% Copyright 2018 Google LLC
%
% Use of this source code is governed by an MIT-style
% license that can be found in the LICENSE file or at
% https://opensource.org/licenses/MIT.

%!TeX spellcheck = en-US

\usepackage[style=alphabetic,backend=biber,alldates=ymd]{biblatex}
\usepackage{tikz}
\usepackage{makecell}
\usepackage{bm}
\usepackage[logic,probability,advantage,adversary,landau,sets,operators]{cryptocode}
\usepackage{algpseudocode}
\usepackage{booktabs}

\usetikzlibrary{positioning}
\usetikzlibrary{groupops}

\providecommand*{\lemmaautorefname}{Lemma}
\renewcommand{\pcadvantagesuperstyle}[1]{#1}

\newcommand*{\parintro}[1]{\textbf{#1}:}

\DeclareMathOperator{\GF}{GF}
\DeclareMathOperator{\XChaCha12}{XChaCha12}
\DeclareMathOperator{\HBSH}{HBSH}
\DeclareMathOperator{\Polydjb}{Poly1305}

\DeclareMathOperator{\NH}{NH}
\DeclareMathOperator{\intify}{int}
\DeclareMathOperator{\fromint}{bin}
\DeclareMathOperator{\pad}{pad}
\DeclareMathOperator{\Perm}{Perm}
\DeclareMathOperator{\LP}{LP}
\DeclareMathOperator{\comp}{\mathsf{comp}}
\DeclareMathOperator{\badQ}{\mathbf{badQ}}
\DeclareMathOperator{\badR}{\mathbf{badR}}
\DeclareMathOperator{\bad}{\mathbf{bad}}

\newcommand*{\Concat}{\Vert}
\newcommand*{\defeq}{\stackrel{\text{def}}{=}}
\newcommand*{\minone}[1]{\min\left(1,#1\right)}

\addbibresource{bib.bib}

\title{{Adiantum}: length-preserving encryption for entry-level processors}

\tikzset{cbox/.style={
        rectangle,
        thick,
        draw,
        minimum height=1cm,
        text centered,
        anchor=center,
        rounded corners=2pt,
    }
}

\usepackage{subfiles}
\def\subbib{\printbibliography}
\begin{document}
\def\subbib{}
\maketitle
\iftoggle{iacr}{
    \keywords{super-pseudorandom permutation \and
        variable input length \and
        tweakable encryption \and
        disk encryption}
}{ % iacr
} % non-iacr

\begin{abstract}
    We present HBSH, a simple construction for tweakable length-preserving encryption which
    directly supports the fastest options for hashing and stream encryption for processors
    without AES or other crypto instructions, with a provable
    quadratic advantage bound. Our composition Adiantum uses NH, Poly1305, XChaCha12,
    and a single AES invocation. On an ARM Cortex-A7 processor, Adiantum decrypts
    4096-byte messages at 10.6 cycles per byte, over five times faster than
    AES-256-XTS, with a constant-time implementation. We also define HPolyC which is
    simpler and has excellent key agility at 13.6 cycles per byte.
\iftoggle{iacr}{
}{

    This paper: \url{https://ia.cr/2018/720} \\
    Source: \url{https://github.com/google/adiantum} \\
    Email: \href{mailto:paulcrowley@google.com,ebiggers@google.com}{\{paulcrowley,ebiggers\}@google.com}
}
\end{abstract}

\iftoggle{iacr}{
}{ % iacr
    \tableofcontents
} % non-iacr

\subfile{introduction.tex}
\subfile{specification.tex}
\subfile{design.tex}
\subfile{performance.tex}
\subfile{proof.tex}
\subfile{hashing.tex}
\printbibliography[heading=bibintoc]
\appendix
\subfile{acks.tex}
\subfile{github.tex}

\vspace*{\fill}
Version: \texttt{\input{work/git.tex}}
\end{document}
