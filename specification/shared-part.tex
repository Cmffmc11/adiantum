% Copyright 2018 Google LLC
%
% Use of this source code is governed by an MIT-style
% license that can be found in the LICENSE file or at
% https://opensource.org/licenses/MIT.

%!TeX spellcheck = en-US

\usepackage[style=alphabetic,backend=biber]{biblatex}
\usepackage{amsmath}
\usepackage{amssymb}
\usepackage[logic,probability,advantage,adversary,landau,sets,operators]{cryptocode}
\usepackage{algpseudocode}

\usepackage{tikz}
\usepackage{makecell}

\usepackage{longtable}

\DeclareMathOperator{\GF}{GF}
\DeclareMathOperator{\XChaCha12}{XChaCha12}
\DeclareMathOperator{\HBSH}{HBSH}
\DeclareMathOperator{\Polydjb}{Poly1305}

\DeclareMathOperator{\NH}{NH}
\DeclareMathOperator{\intify}{int}
\DeclareMathOperator{\fromint}{fromint}
\DeclareMathOperator{\pad}{pad}

\addbibresource{bib.bib}

\usetikzlibrary{positioning}
\usetikzlibrary{groupops}

\newcommand*{\arrowoplus}{\leftarrow\mkern-12mu\oplus}
\newcommand*{\xprm}[2]{\textsf{\ref*{#1}-#2}}
\newcommand*{\xprmtitle}[2]{\textbf{\xprm{#1}{#2}}}
\newcommand*{\calE}{\mathcal{E}}
\newcommand*{\calD}{\mathcal{D}}
\newcommand*{\barE}{\overline{\calE}}
\newcommand*{\barD}{\overline{\calD}}

\title{{Adiantum}: length-preserving encryption for entry-level processors}
\iftoggle{iacr}{
    \author{Paul~Crowley \and Eric~Biggers}
    \institute{Google LLC \\ \email[paulcrowley@google.com,ebiggers@google.com]{{paulcrowley,ebiggers}@google.com}}
}{ % iacr
    \author{Paul~Crowley}
    \author{Eric~Biggers}
    \affil{Google LLC}
} % non-iacr

\tikzset{cbox/.style={
        rectangle,
        thick,
        draw,
        minimum height=1cm,
        text centered,
        anchor=center,
        rounded corners=2pt,
    }
}

\begin{document}
\maketitle
\iftoggle{iacr}{
    \keywords{strong pseudorandom permutation \and
        variable input length \and
        tweakable encryption \and
        disk encryption}
}{ % iacr
} % non-iacr

\begin{abstract}
    We present HBSH, a simple construction for tweakable length-preserving encryption which
    directly supports the fastest options for hashing and stream encryption for processors
    without AES or other crypto instructions, with a provable
    quadratic advantage bound. Our composition Adiantum uses NH, Poly1305, XChaCha12,
    and a single AES invocation. On an ARM Cortex-A7 processor, Adiantum decrypts
    4096-byte messages at 10.6 cycles per byte, over five times faster than
    AES-256-XTS, with a constant-time implementation. We also define HPolyC which is
    simpler and has excellent key agility at 13.6 cycles per byte.
\iftoggle{iacr}{
}{

    This paper: \url{https://ia.cr/2018/720} \\
    Source: \url{https://github.com/google/adiantum} \\
    Email: \href{mailto:paulcrowley@google.com,ebiggers@google.com}{\{paulcrowley,ebiggers\}@google.com}
}
\end{abstract}

% Copyright 2018 Google LLC
%
% Use of this source code is governed by an MIT-style
% license that can be found in the LICENSE file or at
% https://opensource.org/licenses/MIT.

%!TeX spellcheck = en-US

\documentclass[eprint.tex]{subfiles}
\begin{document}
\section{Introduction}
Two aspects of disk encryption make it a challenge for cryptography.  First,
performance is critical; every extra cycle is a worse user experience, and on a mobile device
a reduced battery life.  Second, the ciphertext can be no larger than the plaintext: a sector-sized
read or write to the filesystem must mean a sector-sized read or write to the underlying device,
or performance will again suffer greatly
(as well as, in the case of writes to flash memory, the life of the device).
Nonce reuse is inevitable as there is nowhere to store a varying nonce, and there is no space
for a MAC; thus standard constructions like AES-GCM are not an option and standard notions
of semantic security are unachievable.  The best that can be done under the circumstances is
a ``tweakable super-pseudorandom permutation'': an attacker with access to both encryption
and decryption functions who can choose tweak and plaintext/ciphertext freely is unable to
distinguish it from a family of independent random permutations.

\subsection{History}

Hasty Pudding Cipher~\cite{hpc} was a variable-input-length primitive presented to the AES contest.
A key innovation
was the idea of a ``spice'', which was later formalized as a ``tweak'' in~\cite{tweakable}.
Another tweakable large-block primitive was Mercy~\cite{mercy},
cryptanalyzed in~\cite{mercycryptanalysis}.

\cite{luby-rackoff} (see also~\cite{maurer-luby-rackoff,ppdes})
shows how to construct a pseudorandom permutation using a three-round Feistel
network of pseudorandom functions;
proves that this is not a secure super-pseudorandom permutation (where the adversary
has access to decryption as well as encryption) and that four rounds suffice for this aim.
BEAR and LION~\cite{bearlion} apply this result to an unbalanced Feistel network to build a
large-block cipher from a hash function and a stream cipher (see also BEAST~\cite{beast}).

\cite{fasterlr} shows that a universal function (here called a ``difference concentrator'')
suffices for the first round, which~\cite{NaorReingold} extends to four-round function
to build a super-pseudorandom permutation.

More recently, proposals in this space have focused on the use of
block ciphers. VIL mode~\cite{brvil} is a CBC-MAC based two-pass variable-input-length construction which
is a PRP but not an SPRP. CMC mode~\cite{cmc} is a true SPRP using two passes of the block cipher;
EME mode~\cite{eme} is similar but parallelizable, while
EME* mode~\cite{emestar} extends EME mode to handle blocks that are not a multiple of the block
cipher size. PEP~\cite{pep}, TET~\cite{tet}, and HEH~\cite{heh} have a mixing layer either side of
an ECB layer.

XCB~\cite{xcb} is a block-cipher based unbalanced three-round Feistel network with an
$\epsilon$-almost-XOR-universal hash function for the first and third rounds
(``hash-XOR-hash''),
which uses block
cipher invocations on the narrow side of the network to ensure that the network is an SPRP, rather
than just a PRP; it also introduces a tweak.
HCTR~\cite{hctr,hctr2}, HCH~\cite{hch}, and HMC~\cite{hmc} reduce this to a single
block cipher invocation within the Feistel network.
These proposals require
either two AES invocations, or an AES invocation and two $\GF(2^{128})$ multiplications,
per 128 bits of input.

\subsection{Our contribution}
On the ARM architecture, the ARMv8 Cryptography Extensions include instructions that make
AES and $\GF(2^{128})$ multiplications much more efficient. However,
smartphones designed for developing markets
often use lower-end processors which
don't support these extensions, and as a result there is no existing SPRP construction which performs
acceptably on them.

On such platforms stream ciphers such as ChaCha12~\cite{chacha} significantly
outperform block ciphers in cycles per byte, especially with constant-time implementations.
Similarly, absent specific processor support, hash functions such as NH~\cite{umac2} and
Poly1305 hash~\cite{poly1305} will be much faster
than a $\GF(2^{128})$ polynomial hash. Since these are the operations that act on the bulk of
the data in a disk-sector-sized block, a hash-XOR-hash
mode of operation relying on them should achieve
much improved performance on such platforms.

To this end, we present the HBSH (hash, block cipher, stream cipher, hash)
construction, which generalizes over constructions such as
HCTR and HCH by taking an $\epsilon$-almost-$\Delta$-universal hash function and a
nonce-accepting stream cipher
as components. Based on this construction, our main proposal is Adiantum,
which uses a combination of NH and Poly1305 for the hashing, XChaCha12 for the stream cipher, and
AES for the single blockcipher application. Adiantum:
\begin{itemize}
    \item is a tweakable, variable-input-length, super-pseudorandom permutation
    \item has a security bound quadratic in the number of queries and linear in message length
    \item is highly parallelizable
    \item needs only three passes over the bulk of the data, or
        two if the XOR is combined with the second hash.
\end{itemize}

Without special cases or extra setup, Adiantum handles:
\begin{itemize}
    \item any message and tweak lengths within the allowed range,
    \item varying message and tweak lengths for the same keys.
\end{itemize}

We also describe a simpler proposal, HPolyC, which sacrifices a little speed on large blocks
for simplicity and greater key agility, leaving out the NH hash layer.

The proof of security differs from other hash-XOR-hash modes in three ways. First,
Poly1305 hash is not XOR universal, but universal over $\ZZ/2^{128}\ZZ$,
so for XOR of hash values we substitute addition and subtraction in the appropriate group.
Second, using the XSalsa20 construction~\cite{xsalsa}, we can directly
build a stream cipher which takes a 192-bit nonce to generate a stream, simplifying
the second Feistel operation and associated proof, as well as subkey generation.
Finally, Poly1305 hash has a much weaker security bound than the $\GF(2^{128})$ polynomial hash;
the proof is shaped around ensuring we pay the smallest multiple of this cost we can.

\subsection{Implementation and test vectors}
Implementations in Python, C, and ARMv7 assembly, as well as thousands of test
vectors and the \LaTeX{} source for this paper,  are available from our source code
repository at \url{https://github.com/google/adiantum}.
\end{document}

% Copyright 2018 Google LLC
%
% Use of this source code is governed by an MIT-style
% license that can be found in the LICENSE file or at
% https://opensource.org/licenses/MIT.

%!TeX spellcheck = en-US

\documentclass[eprint.tex]{subfiles}
\begin{document}
\section{Specification}
The HBSH construction is shown in \autoref{finalfig} and \autoref{pseudocode}.
From plaintext $P$ of at least $n$ bits and a tweak $T$,
it generates a ciphertext $C$ of the same length as $P$.
HBSH divides the plaintext into a right-hand block of $n$ bits and a left-hand block with
the remainder of the input, and applies an unbalanced Feistel network.

\begin{figure}
    \begin{floatrow}
        \ffigbox{
            \subfile{finalfig.tex}
        }{
            \caption{HBSH}\label{finalfig}
        }
        \ffigbox{
            \begin{algorithmic}[0]
                \Procedure{HBSHEncrypt}{$T,P$}
                \State $P_L \Concat P_R \gets P$
                \State $P_M \gets P_R \boxplus H_{K_H}(T, P_L)$
                \State $C_M \gets E_{K_E}(P_M)$
                \State $C_L \gets P_L \oplus S_{K_S}(C_M)[0;\abs{P_L}]$
                \State $C_R \gets C_M \boxminus H_{K_H}(T, C_L)$
                \State $C \gets C_L \Concat C_R$
                \State \textbf{return} $C$
                \EndProcedure
            \end{algorithmic}
            \begin{algorithmic}[0]
                \Procedure{HBSHDecrypt}{$T,C$}
                \State $C_L \Concat C_R \gets C$
                \State $C_M \gets C_R \boxplus H_{K_H}(T, C_L)$
                \State $P_L \gets C_L \oplus S_{K_S}(C_M)[0;\abs{C_L}]$
                \State $P_M \gets E_{K_E}^{-1}(C_M)$
                \State $P_R \gets P_M \boxminus H_{K_H}(T, P_L)$
                \State $P \gets P_L \Concat P_R$
                \State \textbf{return} $P$
                \EndProcedure
            \end{algorithmic}
        }{\caption{Pseudocode for HBSH; $P_R$, $P_M$, $C_M$, $C_R$ are $n$ bits long}\label{pseudocode}}
    \end{floatrow}
\end{figure}

\parintro{Notation}
Partial application is implicit; if we define $f: A \times B \rightarrow C$ and
$a \in A$ then $f_a: B \rightarrow C$, and if $f_a^{-1}$ exists then $f_a^{-1}(f_a(b)) = b$.
\begin{itemize}
    \item $\abs{X}$: length of $X \in \bin^{*}$ in bits
    \item $\lambda$: the empty string $\abs{\lambda} = 0$
    \item $\Concat$: bitstring concatenation
    \item $Y[a;l]$: the subsequence of $Y$ of length $l$ starting at $a$.
    \item $\pad_l(X) = X \Concat 0^v$ where $v$ is the least integer $\geq 0$ such that $l$ divides $\abs{X} + v$
    \item $n$, $l_S$: parameters which depend on the
    primitives from which HBSH is built
    \item $\mathcal{T}$: the
    space of tweaks, which depends on the hash function used
    \item $\mathcal{M} = \bigcup_{i=n}^{l_S + n}\bin^i$: the
    space of plaintexts and ciphertexts
    \item $\mathcal{L} = \bigcup_{i=0}^{l_S}\bin^i$, $\mathcal{R} = \bin^n$: messages are processed in two parts,
    $\mathcal{L} \times \mathcal{R}$
    \item $H: \mathcal{K}_H \times \mathcal{T} \times \mathcal{L} \rightarrow \mathcal{R}$: two-argument hash function with keyspace $\mathcal{K}_H$
    \item $\boxplus, \boxminus: \mathcal{R} \times \mathcal{R} \rightarrow \mathcal{R}$: group operations
    \item $E: \mathcal{K}_E \times \mathcal{R} \rightarrow \mathcal{R}$: $n$-bit block cipher with key space $\mathcal{K}_E$
    \item $S: \mathcal{K}_S \times \mathcal{N} \rightarrow \bin^{l_S}$:
    stream cipher with key space $\mathcal{K}_S$
    and nonce space $\mathcal{N}$
    \item $\HBSH : \mathcal{K}_S \times \mathcal{T} \times \mathcal{M} \rightarrow \mathcal{M}$: the HBSH construction takes a key, a tweak, and
    a plaintext, and returns a ciphertext such that $\abs{\HBSH(K_S, T, M)} = \abs{M}$
\end{itemize}
Where we have eg $P_L \Concat P_R \gets P$
with $P \in \mathcal{M}$, $P_L, P_R$ is the unique
pair of elements in $\mathcal{L} \times \mathcal{R}$ such that
$P_L \Concat P_R = P$.

\parintro{Block cipher}
The block cipher $E$
is only invoked once no matter the size of the input, so for disk-sector-sized inputs
its performance isn't critical. Adiantum and HPolyC use AES-256~\cite{AES}, so $n = 128$ and $\mathcal{K}_E = \bin^{256}$.

\parintro{Stream cipher}
$S$
is a stream cipher which takes a key and a nonce and produces a long random stream. In normal use
the nonce is an $n$-bit string, but for key derivation we use the empty string $\lambda$, which
is distinct from all $n$-bit strings; thus $\{\lambda \} \cup \mathcal{R} \subseteq \mathcal{N}$.

Adiantum and HPolyC use the XChaCha12 stream cipher.
The ChaCha~\cite{chacha}
stream ciphers takes a 64-bit nonce, and RFC7539~\cite{RFC7539} proposes
a ChaCha20 variant with a 96-bit nonce, but we need a 128-bit nonce.
The XSalsa20 construction~\cite{xsalsa}
proposed for Salsa20~\cite{salsa20,salsa812} extends the nonce to 192 bits, and
applies straightforwardly to ChaCha~\cite{xchacha,monocypher,libsodiumxchacha}.
We then construct a function that takes a variable-length nonce of up to
191 bits by padding with a 1 followed by zeroes:
$S_{K_S}(C_M) = \XChaCha12_{K_S}(\pad_{192}(C_M \Concat 1))$ and
$\mathcal{N} = \bigcup_{i=0}^{191}\bin^i$.
The maximum output length $l_S = 2^{73}$,
and keyspace $\mathcal{K}_S = \bin^{256}$.

\parintro{Hash}
$H$
is an $\epsilon$-almost-$\Delta$-universal ($\epsilon$A$\Delta$U) function
(as defined in \autoref{eadudef})
yielding a group element represented as an $n$-bit string.
$\boxplus$ represents addition in a group which depends
on the hash function, and $\boxminus$ subtraction.

\begin{sloppypar}
    HPolyC and Adiantum differ only in their choice of hash function. HPolyC is
    based on Poly1305, while Adiantum uses both Poly1305 and NH;
    specifically little-endian $\NH^T[256, 32, 4]$ with a stride of 2 for fast
    vectorization. In both cases, the group used for $\boxplus$ and $\boxminus$ is
    $\ZZ/2^{128}\ZZ$. The value of $\epsilon$ depends on bounds on the input
    lengths.
    We defer full details to \autoref{hashing}.
\end{sloppypar}

\parintro{Key derivation}
HBSH derives $K_E$ and $K_H$ from $K_S$ using a zero-length nonce:
$K_E \Concat K_H \Concat \ldots = S_{K_S}(\lambda)$. An earlier version of this paper
used $K_H \Concat K_E \Concat \ldots = S_{K_S}(\lambda)$ for HPolyC.

\subbib
\end{document}

% Copyright 2018 Google LLC
%
% Use of this source code is governed by an MIT-style
% license that can be found in the LICENSE file or at
% https://opensource.org/licenses/MIT.

%!TeX spellcheck = en-US

\documentclass[eprint.tex]{subfiles}
\begin{document}
\section{Design}
\parintro{Three-pass structure}
Any secure PRP must have a pass that reads all of the plaintext, followed by a pass that modifies
it all. A secure SPRP must have the same property in the reverse direction;
a three-pass structure therefore seems natural.
$\epsilon$-$\Delta$U functions are the fastest options for reading the plaintext in a
cryptographically useful way, and stream ciphers are the fastest options for modifying it.
$\epsilon$-$\Delta$Us
are typically much faster than stream ciphers, and so the hash-XOR-hash structure emerges as
the best option for performance. This structure also has the advantage that it naturally handles
messages in non-round sizes; many VIL modes need extra wrinkles akin to ciphertext stealing
to handle the case where the message is not
a multiple of the block size of the underlying block cipher.

\parintro{Block cipher}
\cite{luby-rackoff} observes that a three-round Feistel network cannot by itself be a secure SPRP;
a simple attack with two plaintexts and one ciphertext distinguishes it. A single block cipher call
in the narrow part of the unbalanced network suffices to frustrate this attack; the
larger the message, the smaller the relative cost of this call. If the plaintext is exactly $n$ bits
long, the stream cipher is not used, and the construction becomes
$C = E_{K_E}(P \boxplus H_{K_H}(T, \lambda)) \boxminus H_{K_H}(T, \lambda)$
as per Subsection 3.1 of \cite{tweakable}.
Compared to HCTR~\cite{hctr} or HCH~\cite{hch}, we sacrifice
symmetry of encryption with decryption in return for
the ability to run the block cipher and stream cipher in parallel when decrypting.
For disk encryption, decryption performance matters most:
reads are more frequent than writes, and reads generally affect user-perceived latency, while
operating systems can usually perform writes asynchronously in the background.

\parintro{Components}
It's unusual for a construction to require more than two distinct primitive components.
More commonly, a hash-XOR-hash mode uses the block cipher to build a stream cipher
(eg using CTR mode~\cite{ctr})
as well as using it directly on the narrow side of the message.
Using XChaCha12 in place of a block cipher affords a significant increase in performance;
however it cannot easily be substituted in the narrow side of the cipher.
\cite{sarkar1,sarkar2,sarkar3,sarkar4} use only an $\epsilon$AXU function
and a stream cipher, and build a hash-XOR-hash SPRP
with a construction that uses a four-round Feistel network over the non-bulk side of the data
broken into two halves. However if we were to build this using XChaCha12,
such a construction would require four extra invocations of ChaCha per message, which would be
a much greater cost than one block cipher invocation.

\parintro{KDM security}
We do not consider an attack model in which derived keys are presented as input.
Length-preserving encryption
which is KDM-secure in the sense of~\cite{kdm} is impossible, since it is trivial for the
adversary to submit a query with a $g$-function
that constructs a plaintext whose ciphertext is all zeroes.
Whether there is a notion of KDM-security that can be
applied in this domain is an open problem. Users must take care to protect the keys from being
included in the input.

\parintro{Stream cipher}
Users are highly sensitive to the performance of disk encryption; an
extra microsecond decrypting the contents of a sector can mean many users
forgoing encryption altogether.
eBACS~\cite{supercop} tests a wide variety of stream ciphers on a wide variety
of architectures; ChaCha12 is consistently one of the
fastest options for the ``armeabi'' (32-bit ARM) architecture.
ChaCha and its predecessor Salsa20
have seen intense cryptanalysis in the decade or so since publication
\cite{tdcs20,nonrandomsalsa,tsunoo,latindance,ishiguro2011,ishiguro2012,zhenqing2012,
maitra2015,chachamaitra,choudhuri2016,dey2017,Choudhuri_Maitra_2017,chacha2018};
the best attack breaks 7 rounds, a landmark reached with \cite{latindance} in 2008.
Each round greatly increases the difficulty of attack.
We therefore feel confident selecting the 12-round variant as giving
good confidence in security while minimizing the cost to users.

\parintro{Hash function}
Since the $\epsilon$-$\Delta$U is run twice over the bulk of the message, its speed is especially
crucial for large messages. One of the fastest such functions in software is NH, and
it's also appealingly simple; however as discussed in~\autoref{nh} it generally has to be
combined with a second hashing stage, and for this purpose we use Poly1305. The 1KiB input size
we specify for NH means that a simple, portable implementation of Poly1305 can be
used without a great cost in speed; in contrast, for HPolyC a vectorized
Poly1305 implementation is important.
We considered using UHASH (as defined for UMAC~\cite{rfc4418}) rather than our
custom combination of NH and Poly1305; however, available UHASH implementations
are not constant-time, and a constant-time implementation would be significantly
slower.

\parintro{Key agility}
For the 4KiB messages of disk encryption,
the ~1KiB NH key size has only a small impact on key agility. Applications
that need high key agility even on small messages may instead use HPolyC, which
uses Poly1305 directly. The main
cost of a new HPolyC key is a single XChaCha12 invocation to generate subkeys.
ChaCha12 has no key schedule
and makes no use of precomputation; XChaCha12
requires one extra call to the ChaCha permutation for each new nonce.
No extra work is required for differing message or tweak lengths for either Adiantum
or HPolyC.

\parintro{Constant-time}
NH, Poly1305 and ChaCha12 are designed such that the most natural fast implementations are
constant-time and free from data-dependent lookups. So long as the block cipher implementation
also has these properties, Adiantum and HPolyC will inherit security against
this class of side-channel attacks.

\parintro{Parallelizability}
NH, Poly1305 and ChaCha12 are highly parallelizable.
The stream cipher and second hash stages can also be run in combination for a total
of two passes over the bulk of the data, unlike a mode such as HEH~\cite{heh}
which requires at least three.
We put the ``special'' part on the right so that in typical uses the bulk encryption has
the best alignment for fast operations.

\parintro{Naming}
``Adiantum'' is the genus of the maidenhair fern, which in the language of
flowers (floriography) signifies sincerity and discretion.~\cite{fleurs}

\subbib
\end{document}

% Copyright 2018 Google LLC
%
% Use of this source code is governed by an MIT-style
% license that can be found in the LICENSE file or at
% https://opensource.org/licenses/MIT.

%!TeX spellcheck = en-US

\documentclass[eprint.tex]{subfiles}
\begin{document}
\section{Performance}\label{performance}

In \autoref{performancetable} we
show performance on an ARM \mbox{Cortex-A7}
processor in the Snapdragon 400 chipset running at \mbox{1.19 GHz}.  This
processor supports the NEON vector instruction set, but not the ARM cryptographic
extensions; it is used in many smartphones and smartwatches, especially low-end
devices, and is representative of the kind of platform we mean to target.
Where the figures are within 2\%, a single row is shown for both encryption and
decryption.

\begin{table}
    \caption{Performance on ARM Cortex-A7}
    \label{performancetable}
    \centering
    \begin{tabular}{lrr}
        Algorithm &
            \makecell{Cycles per byte \\ (4096-byte sectors)} &
            \makecell{Cycles per byte \\ (512-byte sectors)} \\
    \hline
    % Copyright 2018 Google LLC
%
% Use of this source code is governed by an MIT-style
% license that can be found in the LICENSE file or at
% https://opensource.org/licenses/MIT.

%!TeX spellcheck = en-US

\documentclass[eprint.tex]{subfiles}
\begin{document}
\section{Performance}\label{performance}

In \autoref{performancetable} we
show performance on an ARM \mbox{Cortex-A7}
processor in the Snapdragon 400 chipset running at \mbox{1.19 GHz}.  This
processor supports the NEON vector instruction set, but not the ARM cryptographic
extensions; it is used in many smartphones and smartwatches, especially low-end
devices, and is representative of the kind of platform we mean to target.
Where the figures are within 2\%, a single row is shown for both encryption and
decryption.

\begin{table}
    \caption{Performance on ARM Cortex-A7}
    \label{performancetable}
    \centering
    \begin{tabular}{lrr}
        Algorithm &
            \makecell{Cycles per byte \\ (4096-byte sectors)} &
            \makecell{Cycles per byte \\ (512-byte sectors)} \\
    \hline
    % Copyright 2018 Google LLC
%
% Use of this source code is governed by an MIT-style
% license that can be found in the LICENSE file or at
% https://opensource.org/licenses/MIT.

%!TeX spellcheck = en-US

\documentclass[eprint.tex]{subfiles}
\begin{document}
\section{Performance}\label{performance}

In \autoref{performancetable} we
show performance on an ARM \mbox{Cortex-A7}
processor in the Snapdragon 400 chipset running at \mbox{1.19 GHz}.  This
processor supports the NEON vector instruction set, but not the ARM cryptographic
extensions; it is used in many smartphones and smartwatches, especially low-end
devices, and is representative of the kind of platform we mean to target.
Where the figures are within 2\%, a single row is shown for both encryption and
decryption.

\begin{table}
    \caption{Performance on ARM Cortex-A7}
    \label{performancetable}
    \centering
    \begin{tabular}{lrr}
        Algorithm &
            \makecell{Cycles per byte \\ (4096-byte sectors)} &
            \makecell{Cycles per byte \\ (512-byte sectors)} \\
    \hline
    \input{work/performance.tex}
    \end{tabular}
\end{table}

We have prioritized performance on 4096-byte messages, but we also tested 512-byte messages.
512-byte disk sectors were the standard until the introduction of Advanced Format in 2010;
modern large hard drives and flash drives now use 4096-byte sectors.
On Linux, 4096 bytes is the standard page size, the standard allocation unit size for filesystems,
and the granularity of \textit{fscrypt} file-based encryption, while
\mbox{\textit{dm-crypt}} full-disk encryption has recently been updated
to support this size.

For comparison we evaluate against various block ciphers in XTS mode~\cite{xts}:
AES\mbox~\cite{AES}, Speck~\cite{speck1,speck2,speck3}, NOEKEON~\cite{noekeon},
and XTEA~\cite{xtea}. We also include the performance of ChaCha, NH,
and Poly1305 by themselves for reference.

We used the fastest constant-time implementation of each algorithm we were able
to find or write for the platform; see \autoref{implementation}.  As an
exception, given the high difficulty of writing truly constant-time AES
software~\cite{aes-cache-timing}, for single-block AES we tolerate an
implementation that merely prefetches the lookup tables as a hardening measure.
In every case the performance-critical parts were written in assembly language,
usually using NEON instructions.  Our tests complete processing of each message
before starting the next, so latency of a single message in cycles is the
product of message size and cpb.

\begin{table}
    \caption{Implementations}
    \label{implementation}
    \centering
    \begin{tabular}{llp{7cm}}
        Algorithm & Source & Notes \\
    \hline
    ChaCha & Linux v4.17 & \texttt{chacha20-neon-core.S}, modified to support
        ChaCha8 and ChaCha12; also applied optimizations from \texttt{cryptodev}
        commit a1b22a5f45fe8841 \\
    Poly1305 & OpenSSL 1.1.0h & \texttt{poly1305-armv4.S}, modified to
        precompute key powers just once per key \\
    \mbox{AES} & Linux v4.17 & \texttt{aes-cipher-core.S}, modified to prefetch
        lookup tables \\
    \mbox{AES-XTS} & Linux v4.17 & \texttt{aes-neonbs-core.S} (bit-sliced) \\
    \mbox{Speck128/256-XTS} & Linux v4.17 & \texttt{speck-neon-core.S} \\
    \mbox{NOEKEON-XTS} & ours & \\
    \mbox{XTEA-XTS} & ours & \\
    \end{tabular}
\end{table}

Adiantum and HPolyC are the only algorithms in \autoref{performancetable} that are tweakable
super-pseudorandom permutations over the entire sector.  We expect any AES-based
construction to that end to be significantly slower than \mbox{AES-XTS}.

We conclude that for 4096-byte sectors, Adiantum
(aka \mbox{Adiantum-XChaCha12-AES}) can perform
significantly better than an aggressively designed block cipher (\mbox{Speck128/256}) in XTS mode.
Efficient implementations of NH, Poly1305 and ChaCha are available for many
platforms, as these algorithms are well-suited for implementation with either
general-purpose scalar instructions or with general-purpose vector instructions
such as NEON or AVX2.

For a greater margin of security at a slower speed, ChaCha20 can
be used instead of ChaCha12; the same stream cipher
must be used for key derivation as for the Feistel function. Similarly, one could substitute
NOEKEON in place of AES-256 to make defense against timing attacks easier and improve performance.
This may weaken security against a brute-force attack since NOEKEON has only a 128-bit key, though
it's not obvious how to mount such an attack when the hashing and stream cipher keys are unknown.
Note that this is a different axis of security than success probability; an
attack that needs (say) $2^{40}$ work and always succeeds is a much bigger
problem than than an attack that needs negligible work and succeeds with
probability $2^{-40}$.

\subbib
\end{document}

    \end{tabular}
\end{table}

We have prioritized performance on 4096-byte messages, but we also tested 512-byte messages.
512-byte disk sectors were the standard until the introduction of Advanced Format in 2010;
modern large hard drives and flash drives now use 4096-byte sectors.
On Linux, 4096 bytes is the standard page size, the standard allocation unit size for filesystems,
and the granularity of \textit{fscrypt} file-based encryption, while
\mbox{\textit{dm-crypt}} full-disk encryption has recently been updated
to support this size.

For comparison we evaluate against various block ciphers in XTS mode~\cite{xts}:
AES\mbox~\cite{AES}, Speck~\cite{speck1,speck2,speck3}, NOEKEON~\cite{noekeon},
and XTEA~\cite{xtea}. We also include the performance of ChaCha, NH,
and Poly1305 by themselves for reference.

We used the fastest constant-time implementation of each algorithm we were able
to find or write for the platform; see \autoref{implementation}.  As an
exception, given the high difficulty of writing truly constant-time AES
software~\cite{aes-cache-timing}, for single-block AES we tolerate an
implementation that merely prefetches the lookup tables as a hardening measure.
In every case the performance-critical parts were written in assembly language,
usually using NEON instructions.  Our tests complete processing of each message
before starting the next, so latency of a single message in cycles is the
product of message size and cpb.

\begin{table}
    \caption{Implementations}
    \label{implementation}
    \centering
    \begin{tabular}{llp{7cm}}
        Algorithm & Source & Notes \\
    \hline
    ChaCha & Linux v4.17 & \texttt{chacha20-neon-core.S}, modified to support
        ChaCha8 and ChaCha12; also applied optimizations from \texttt{cryptodev}
        commit a1b22a5f45fe8841 \\
    Poly1305 & OpenSSL 1.1.0h & \texttt{poly1305-armv4.S}, modified to
        precompute key powers just once per key \\
    \mbox{AES} & Linux v4.17 & \texttt{aes-cipher-core.S}, modified to prefetch
        lookup tables \\
    \mbox{AES-XTS} & Linux v4.17 & \texttt{aes-neonbs-core.S} (bit-sliced) \\
    \mbox{Speck128/256-XTS} & Linux v4.17 & \texttt{speck-neon-core.S} \\
    \mbox{NOEKEON-XTS} & ours & \\
    \mbox{XTEA-XTS} & ours & \\
    \end{tabular}
\end{table}

Adiantum and HPolyC are the only algorithms in \autoref{performancetable} that are tweakable
super-pseudorandom permutations over the entire sector.  We expect any AES-based
construction to that end to be significantly slower than \mbox{AES-XTS}.

We conclude that for 4096-byte sectors, Adiantum
(aka \mbox{Adiantum-XChaCha12-AES}) can perform
significantly better than an aggressively designed block cipher (\mbox{Speck128/256}) in XTS mode.
Efficient implementations of NH, Poly1305 and ChaCha are available for many
platforms, as these algorithms are well-suited for implementation with either
general-purpose scalar instructions or with general-purpose vector instructions
such as NEON or AVX2.

For a greater margin of security at a slower speed, ChaCha20 can
be used instead of ChaCha12; the same stream cipher
must be used for key derivation as for the Feistel function. Similarly, one could substitute
NOEKEON in place of AES-256 to make defense against timing attacks easier and improve performance.
This may weaken security against a brute-force attack since NOEKEON has only a 128-bit key, though
it's not obvious how to mount such an attack when the hashing and stream cipher keys are unknown.
Note that this is a different axis of security than success probability; an
attack that needs (say) $2^{40}$ work and always succeeds is a much bigger
problem than than an attack that needs negligible work and succeeds with
probability $2^{-40}$.

\subbib
\end{document}

    \end{tabular}
\end{table}

We have prioritized performance on 4096-byte messages, but we also tested 512-byte messages.
512-byte disk sectors were the standard until the introduction of Advanced Format in 2010;
modern large hard drives and flash drives now use 4096-byte sectors.
On Linux, 4096 bytes is the standard page size, the standard allocation unit size for filesystems,
and the granularity of \textit{fscrypt} file-based encryption, while
\mbox{\textit{dm-crypt}} full-disk encryption has recently been updated
to support this size.

For comparison we evaluate against various block ciphers in XTS mode~\cite{xts}:
AES\mbox~\cite{AES}, Speck~\cite{speck1,speck2,speck3}, NOEKEON~\cite{noekeon},
and XTEA~\cite{xtea}. We also include the performance of ChaCha, NH,
and Poly1305 by themselves for reference.

We used the fastest constant-time implementation of each algorithm we were able
to find or write for the platform; see \autoref{implementation}.  As an
exception, given the high difficulty of writing truly constant-time AES
software~\cite{aes-cache-timing}, for single-block AES we tolerate an
implementation that merely prefetches the lookup tables as a hardening measure.
In every case the performance-critical parts were written in assembly language,
usually using NEON instructions.  Our tests complete processing of each message
before starting the next, so latency of a single message in cycles is the
product of message size and cpb.

\begin{table}
    \caption{Implementations}
    \label{implementation}
    \centering
    \begin{tabular}{llp{7cm}}
        Algorithm & Source & Notes \\
    \hline
    ChaCha & Linux v4.17 & \texttt{chacha20-neon-core.S}, modified to support
        ChaCha8 and ChaCha12; also applied optimizations from \texttt{cryptodev}
        commit a1b22a5f45fe8841 \\
    Poly1305 & OpenSSL 1.1.0h & \texttt{poly1305-armv4.S}, modified to
        precompute key powers just once per key \\
    \mbox{AES} & Linux v4.17 & \texttt{aes-cipher-core.S}, modified to prefetch
        lookup tables \\
    \mbox{AES-XTS} & Linux v4.17 & \texttt{aes-neonbs-core.S} (bit-sliced) \\
    \mbox{Speck128/256-XTS} & Linux v4.17 & \texttt{speck-neon-core.S} \\
    \mbox{NOEKEON-XTS} & ours & \\
    \mbox{XTEA-XTS} & ours & \\
    \end{tabular}
\end{table}

Adiantum and HPolyC are the only algorithms in \autoref{performancetable} that are tweakable
super-pseudorandom permutations over the entire sector.  We expect any AES-based
construction to that end to be significantly slower than \mbox{AES-XTS}.

We conclude that for 4096-byte sectors, Adiantum
(aka \mbox{Adiantum-XChaCha12-AES}) can perform
significantly better than an aggressively designed block cipher (\mbox{Speck128/256}) in XTS mode.
Efficient implementations of NH, Poly1305 and ChaCha are available for many
platforms, as these algorithms are well-suited for implementation with either
general-purpose scalar instructions or with general-purpose vector instructions
such as NEON or AVX2.

For a greater margin of security at a slower speed, ChaCha20 can
be used instead of ChaCha12; the same stream cipher
must be used for key derivation as for the Feistel function. Similarly, one could substitute
NOEKEON in place of AES-256 to make defense against timing attacks easier and improve performance.
This may weaken security against a brute-force attack since NOEKEON has only a 128-bit key, though
it's not obvious how to mount such an attack when the hashing and stream cipher keys are unknown.
Note that this is a different axis of security than success probability; an
attack that needs (say) $2^{40}$ work and always succeeds is a much bigger
problem than than an attack that needs negligible work and succeeds with
probability $2^{-40}$.

\subbib
\end{document}

% Copyright 2018 Google LLC
%
% Use of this source code is governed by an MIT-style
% license that can be found in the LICENSE file or at
% https://opensource.org/licenses/MIT.

%!TeX spellcheck = en-US

\documentclass[eprint.tex]{subfiles}
\begin{document}
\section{Security reduction}
\subsection{Definitions}
Below we define the security properties of HBSH and
the primitives it uses, and prove a relationship the two.
In what follows we draw on definitions used in \cite{hctr2}.

\parintro{Hash function}
The hash function
$H: \mathcal{K}_H \times \mathcal{T} \times \mathcal{L} \rightarrow \bin^n$
must be $\epsilon$-almost-$\Delta$-universal\label{eadudef} for some $\epsilon$:
for any $g \in \bin^n$ and
any two distinct messages $(T, L) \neq (T', L')$:
%
\begin{displaymath}
\probsub{K \sample \mathcal{K}_H}{H_{K}(T, L) \boxminus H_{K}(T', L') = g} \leq \epsilon
\end{displaymath}
%
For Adiantum and HPolyC the value of $\epsilon$ will depend on bounds on the lengths of $T$ and $L$.

\parintro{Stream cipher}
The stream cipher
$S: \mathcal{K}_S \times \mathcal{N} \rightarrow \bin^{l_S}$
must be a pseudorandom function.
%
\begin{align*}
    \advantage{\mathrm{prf}}{S}[(A)] \defeq
    &\left| \probsub{K \sample \mathcal{K}_S}{A^{S_{K}}\Rightarrow 1}\right.
    \\
    &\left. - \probsub{F \sample (\mathcal{N} \rightarrow \bin^{l_S})}
    {A^F\Rightarrow 1} \right|
    \\
    \advantage{\mathrm{prf}}{S}[(q, l, t)]
    \defeq &\max_{A \in \mathcal{A}(q, l, t)} \advantage{\mathrm{prf}}{S}[(A)]
\end{align*}
%
where $A$ is an adversary,
$\mathcal{N} \rightarrow \bin^{l_S}$ denotes the set of all
functions from $\mathcal{N}$ to $\bin^{l_S}$,
and
$\mathcal{A}(q, l, t)$
is the set of all adversaries that make at most $q$ queries, discard all but $l$ bits from
the results of those queries, and take at most $t$ time.

\parintro{Block cipher}
The block cipher
$E: \mathcal{K}_E \times \bin^n \rightarrow \bin^n$
must be a super-pseudorandom permutation.
%
\begin{align*}
    \advantage{\pm \mathrm{prp}}{E}[(A)] \defeq
    &\left|\probsub{K \sample \mathcal{K}_E}{A^{E_K,E_K^{-1}}\Rightarrow 1}\right.
    \\
    &\left. - \probsub{\pi \sample \Perm(\bin^n)}{A^{\pi,\pi^{-1}}\Rightarrow 1}\right|
    \\
    \advantage{\pm \mathrm{prp}}{E}[(q, t)] \defeq
    &\max_{A \in \mathcal{A}(q, t)} \advantage{\pm \mathrm{prp}}{E}[(A)]
\end{align*}
%
where $A$ is an adversary,
$\Perm(S)$ denotes the set of all permutations on a set $S$,
and
$\mathcal{A}(q, t)$
is the set of all adversaries that make at most $q$ queries and take at most $t$ time.

\parintro{HBSH}
HBSH itself is a tweakable length-preserving encryption system.
Let $\LP^\mathcal{T}(\mathcal{M})$ denote the set of all
tweakable length-preserving functions
$\bm{f} : \mathcal{T} \times \mathcal{M} \rightarrow \mathcal{M}$
such that for all $T, M \in \mathcal{T} \times \mathcal{M}$,
$|\bm{f}(T, M)| = |M|$. Let $\Perm^\mathcal{T}(\mathcal{M})$ denote
the subset of $\LP^\mathcal{T}(\mathcal{M})$ such that
for all $T \in \mathcal{T}$, $\bm{f}_{T}$ is a bijection.
In an abuse of notation,
where $\bm{\pi}$ is a member of this set,
we use $\bm{\pi}^{-1}$ to refer to the function
such that $\bm{\pi}^{-1}(T, \bm{\pi}(T, M)) = M$ ie $(\bm{\pi}^{-1})_T = (\bm{\pi}_T)^{-1}$.

For a tweakable length-preserving encryption system
$\bm{E} : \mathcal{K} \times \mathcal{T} \times \mathcal{M} \rightarrow \mathcal{M}$
the distinguishing advantage of an adversary $A$ is:
%
\begin{align*}
    \advantage{\pm \widetilde{\mathrm{prp}}}{\bm{E}}[(A)] \defeq
    &\left|\probsub{K \sample \mathcal{K}}{A^{\bm{E}_K,\bm{E}_K^{-1}}\Rightarrow 1}\right.
    \\
    &\left. - \probsub{\bm{\pi} \sample \Perm^\mathcal{T}(\mathcal{M})}
        {A^{\bm{\pi},\bm{\pi}^{-1}}\Rightarrow 1}\right|
    \\
\intertext{and}
\advantage{\pm \widetilde{\mathrm{prp}}}{\bm{E}}[(q, l_T, l_M, t)]
\defeq &
\max_{A \in \mathcal{A}(q, l_T, l_M, t)} \advantage{\pm \widetilde{\mathrm{prp}}}{\bm{E}}[(A)]
\end{align*}
%
\begin{displaymath}
\end{displaymath}
where $\mathcal{A}(q, l_T, l_M, t)$
is the set of all adversaries that
make at most $q$ queries
with tweak of length at most $l_T$
and message of length at most $l_M$
and take at most $t$ time.

\subsection{Primary claim}
\begin{theorem}\label{hbshadvantage}
    Where HBSH mode is instantiated with hash function $H$, block cipher $E$ and stream cipher $S$,
    and where $H$ is $\epsilon$-almost-$\Delta$-universal for inputs such that
    $|T| \leq l_T$, $|L| \leq l_M - n$, then:
    %
    \begin{align*}
        \advantage{\pm \widetilde{\mathrm{prp}}}{\HBSH}[(q, l_T, l_M, t)]
        \leq &(\epsilon + 2(2^{-n}))\binom{q}{2} \\
        &+ \advantage{\mathrm{prf}}{S}[(q + 1, |K_E| + |K_H| + q(l_M - n), t')] \\
        &+ \advantage{\pm \mathrm{prp}}{E}[(q, t')]\\
    \end{align*}
    %
    where $t' \defeq t + \bigO{q(l_T + l_M)}$.
\end{theorem}

\begin{proof}Deferred to \autoref{hbshproof}.\renewcommand{\qedsymbol}{}
\end{proof}

\subsection{Preliminaries}
\parintro{H-coefficient technique}
We begin by using the ``H-coefficient'' technique to prove a distinguishing bound
between random query responses and
an ``idealized'' version of HBSH that uses a random function and permutation
in place of pseudorandom primitives.

The H-coefficient technique was introduced by Patarin in 1991~\cite{ppdes,hco};
for definitions
of the terms, symbols, and inequalities we rely on here, see \cite{hco2} Section 3,
``The H-coefficient Technique in a Nutshell''.
As per that description, WLOG we assume a deterministic adversary $A$. An oracle $\omega$ is
sampled from a distribution of oracles, either $\Omega_X$ or
$\Omega_Y$ as appropriate; the transcript $\tau$ is then completely determined by the combination
of attacker and probabilistically chosen deterministic oracle.

\parintro{Transcript}
Our transcript $\tau$ is a sequence of tuples
$(d^i, T^i, P^i, C^i)$
in
$\{+, -\} \times \mathcal{T} \times \mathcal{M} \times \mathcal{M}$
for $i \in [0 \ldots q-1]$.
For the $i$th sequential query
$d^i$ is the direction of the query:
if $d^i = +$ then the left oracle is queried with $T^i, P^i$ and the result is $C^i$,
while if $d^i = -$ then the right oracle is queried with $T^i, C^i$ and the result is $P^i$.

\parintro{Allowed queries}
We consider adversaries contained in $\mathcal{A}(q, l_T, l_M, t)$ for some value of
the bounds $q$, $l_T$, $l_M$, $t$.
Without loss of generality, we consider only adversaries who do not make ``pointless''
queries as defined in \cite{cmc}. Thus for $i < j$, if $d^j = +$ then
$(T^j, P^j) \neq (T^i, P^i)$, and similarly if $d^j = -$ then
$(T^j, C^j) \neq (T^i, C^i)$.

\parintro{Helper functions}
We define here
helper functions $\xi$, $\theta$, $\phi$, and $\eta$, useful for constructing
HBSH-like ciphers. Where a parameter is given as
$L || R$, $|R|=n$.

\begin{align*}
    \xi :& \mathcal{K}_H \times \mathcal{T} \times \mathcal{M} \rightarrow \bin^n \\
    \xi_{K_H, T}(L||R) \defeq& R \boxplus H_{K_H}(T, L) \\
    \vphantom{|} \\
    \phi :& \mathcal{K}_H \times \mathcal{T} \times \mathcal{M} \rightarrow \mathcal{M} \\
    \phi_{K_H, T}(L || R) \defeq& L || \xi_{K_H, T}(L||R) \\
    =& L || (R \boxplus H_{K_H}(T, L)) \\
    \phi^{-1}_{K_H, T}(L || R) =& L || (R \boxminus H_{K_H}(T, L)) \\
    \vphantom{|} \\
    \theta :& \Perm(\bin^n) \times (\mathcal{N} \rightarrow \bin^{l_S}) \times \mathcal{M} \rightarrow \mathcal{M} \\
    \theta_{\pi, F}(L || R) \defeq& (L \arrowoplus F(\pi(R))) || \pi(R) \\
    \vphantom{|} \\
    \eta :& \mathcal{K}_H \times \Perm(\bin^n) \times (\mathcal{N} \rightarrow \bin^{l_S}) \times \mathcal{T} \times \mathcal{M} \rightarrow \mathcal{M} \\
    \eta_{K_H, \pi, F, T} \defeq& \phi_{K_H,T}^{-1} \circ \theta_{\pi, F} \circ \phi_{K_H,T} \\
\end{align*}

\parintro{Ideal world}
Our ``ideal world''
is a pair of length-preserving functions
$\omega = \mathcal{E}, \mathcal{D}$
sampled fairly from
$\Omega_Y \defeq \LP^\mathcal{T}(\mathcal{M}) \times \LP^\mathcal{T}(\mathcal{M})$.
$Y$ is a random variable representing the distribution of transcripts for
$A^{\mathcal{E}, \mathcal{D}}$.

\parintro{Real world}
Our ``real world'' is an idealized form of HBSH which uses a random function and permutation:
$\omega = K_H, \pi, F$
sampled fairly from
$\Omega_X \defeq \mathcal{K}_H \times \Perm(\bin^n) \times (\mathcal{N} \rightarrow \bin^{l_S})$.
$X$ is a random variable representing the distribution of transcripts for
$A^{\eta_{\omega}, \eta_{\omega}^{-1}}$.

\parintro{Bad events}
We define two bad events $\badQ$ and $\badR$.

\begin{itemize}
    \item $(K_H, \tau) \in \badQ$ if there exists $i < j$ such that
    \begin{itemize}
        \item either $d^j = +$ and $\xi(K_H, T^i, P^i) = \xi(K_H, T^j, P^j)$
        \item or $d^j = -$ and $\xi(K_H, T^i, C^i) = \xi(K_H, T^j, C^j)$.
    \end{itemize}
    \item $(K_H, \tau) \in \badR$ if there exists $i < j$ such that
    \begin{itemize}
        \item either $d^j = -$ and $\xi(K_H, T^i, P^i) = \xi(K_H, T^j, P^j)$
        \item or $d^j = +$ and $\xi(K_H, T^i, C^i) = \xi(K_H, T^j, C^j)$.
    \end{itemize}
\end{itemize}

Finally we define the disjunction
$\bad \defeq \badQ \cup \badR$.

\subsection{Lemmas}
\begin{lemma} \label{badQ}
    For any $\tau$ such that $\prob{Y = \tau} > 0$,
    \begin{displaymath}
        \probsub{K_H \sample \mathcal{K}_H}{(K_H, \tau) \in \badQ}
        \leq \epsilon \binom{q}{2}
    \end{displaymath}
\end{lemma}

\begin{proof}
Assume $d^j = +$ for some pair $i, j$, and let $L^i || R^i = P^i$ and similarly for $P^j$.
From $\prob{Y = \tau} > 0$ we know that $|T^i|, |T^j| \leq l_T$ and $|P^i|, |P^j| \leq l_M$,
and therefore that $|L^i|, |L^j| \leq l_M - n$.
Because pointless queries are forbidden we also know that $(T^i, P^i) \neq (T^j, P^j)$.
If $(T^i, L^i) = (T^j, L^j)$ then $R^i \neq R^j$ and equality cannot occur.
Otherwise by the $\epsilon$A$\Delta$U property of $H$ this occurs with probability
at most $\epsilon$:

\begin{align*}
    &\xi(K_H, T^i, L^i||R^i) = \xi(K_H, T^j, L^j||R^j) \\
    \Leftrightarrow& R^i \boxplus H_{K_H}(T^i, L^i) = R^j \boxplus H_{K_H}(T^j, L^j) \\
    \Leftrightarrow& H_{K_H}(T^i, L^i) \boxminus H_{K_H}(T^j, L^j) = R^j \boxminus R^i \\
\end{align*}

Where $d^j = -$, a similar argument applies for $C^i$, $C^j$.
For an upper bound, we sum across all $\binom{q}{2}$ pairs $i, j$.
\end{proof}

\begin{lemma} \label{badR}
    For any $K_H \sample \mathcal{K}_H$,
    \begin{displaymath}
        \probsub{\tau \sim Y}{(K_H, \tau) \in \badR}
        \leq 2^{-n} \binom{q}{2}
    \end{displaymath}
\end{lemma}

\begin{proof}
    Assume $d^j = +$ for some pair $i, j$, and let $L^i || R^i = C^i$ and similarly for $C^j$.
    Because pointless queries are forbidden, in $Y$ world,
    conditioning on all prior queries and responses,
    all possible values of $C^j$ such that $|C^j| = |P^j|$ will be equally likely.
    In particular, all values of $R^j$ are equally likely. Therefore
    $\prob{R^j = R^i \boxplus H_{K_H}(T^i, L^i) \boxminus H_{K_H}(T^j, L^j)} = 2^{-n}$.

    Where $d^j = -$, a similar argument applies for $P^i$, $P^j$.
    For an upper bound, we sum across all $\binom{q}{2}$ pairs $i, j$.
\end{proof}

\begin{lemma} \label{notbad}
    For any $K_H \in \mathcal{K}_H$ and transcript $\tau$ such that $\prob{Y = \tau} > 0$ and
    $(K_H, \tau) \notin \bad$,
    \begin{displaymath}
        \condprobsub{\Omega_X}{\omega \in \comp_X(\tau)}{\omega = (K_H, ., .)}
        \geq
        \probsub{\Omega_Y}{\omega \in \comp_Y(\tau)}
    \end{displaymath}
\end{lemma}

\begin{proof}
    In the $Y$ world, for any transcript such that $\prob{Y = \tau} > 0$,
    since all queries are distinct, the responses are independent of all
    previous responses, and
    $\probsub{\Omega_Y}{\omega \in \comp_Y(\tau)} = \prod_i 2^{-|P^i|}$.
    Let $P_L^i || P_R^i = P^i$, $P_M^i = \xi_{K_H, T^i}(P^i)$ and similarly for $C^i$.
    Since $(K_H, \tau) \notin \bad$ we have that $P_M^i \neq P_M^j$
    and $C_M^i \neq C_M^j$ for all $i \neq j$.

    \begin{align*}
        & \eta_{\omega, T^i}(P^i) = C^i\\
        \Leftrightarrow & \phi_{K_H,T^i}^{-1}(\theta_{\pi, F}(\phi_{K_H,T^i}(P^i))) = C^i\\
        \Leftrightarrow & \theta_{\pi, F}(P_L^i || P_M^i) = C_L^i || C_M^i \\
        \Leftrightarrow & \pi(P_M^i) = C_M^i \wedge F(C_M^i)[0;|P^i| - n] = P_L^i \oplus C_L^i \\
    \end{align*}

    These conditions are independent, since they depend on independently drawn
    variables:
    \begin{displaymath}
        \probsub{
            F \sample (\mathcal{N} \rightarrow \bin^{l_S})
        }{
            \forall_i : F(C_M^i)[0;|P^i| - n] = P_L^i \oplus C_L^i
        } = \prod_i 2^{-(|P^i| - n)}
    \end{displaymath}
    and
    \begin{displaymath}
        \probsub{
            \pi \sample \Perm(\bin^n)
        }{
            \forall_i : \pi(P_M^i) = C_M^i
        } = \prod_i \frac{1}{2^n - i}
    \end{displaymath}

    Therefore:
    \begin{align*}
        &\condprobsub{\Omega_X}{\omega \in \comp_X(\tau)}{\omega = (K_H, ., .)} \\
        =& \probsub{
            \pi \sample \Perm(\bin^n),
            F \sample (\mathcal{N} \rightarrow \bin^{l_S})
        }{
            \forall_i : \eta_{\omega, T^i}(P^i) = C^i
        } \\
        =& \prod_i \frac{1}{2^n - i}2^{-(|P^i| - n)} \\
        \geq & \prod_i 2^{-|P^i|} = \probsub{\Omega_Y}{\omega \in \comp_Y(\tau)}\\
    \end{align*}
\end{proof}

\begin{lemma} \label{xyadv}
    \begin{displaymath}
        \advantage{\pm \widetilde{\mathrm{rnd}}}{\eta}[(q, l_T, l_M, t)]
        \leq (\epsilon + 2^{-n})\binom{q}{2}
    \end{displaymath}
\end{lemma}

\begin{proof}
    $\advantage{\pm \widetilde{\mathrm{rnd}}}{\eta}[(q, l_T, l_M, t)]
    = \max_{A \in \mathcal{A}(q, l_T, l_M, t)} |\rho_X - \rho_Y|$ where
    \begin{align*}
        \rho_X \defeq& \probsub{\omega \sample \Omega_X}{A^{\eta_{\omega}, \eta_{\omega}^{-1}}\Rightarrow 1} \\
        \rho_Y \defeq& \probsub{\mathcal{E}, \mathcal{D} \sample \Omega_Y}{A^{\mathcal{E}, \mathcal{D}}\Rightarrow 1} \\
    \end{align*}

    Applying the H-coefficient technique:
    \begin{align*}
        &|\rho_X - \rho_Y| \\
        \leq& 1 - \expsub{\tau \sim Y}{\min
            \left(1,
               \frac{\probsub{\Omega_X}{\omega \in \comp_X(\tau)}}
               {\probsub{\Omega_Y}{\omega \in \comp_Y(\tau)}}
            \right)} \\
        =& 1 - \expsub{\tau \sim Y}{\min
            \left(1, \sum_{K_H \in \mathcal{K}_H}
              \frac{\probsub{\Omega_X}{\omega \in \comp_X(\tau) \wedge \omega = (K_H, ., .)}}
              {\probsub{\Omega_Y}{\omega \in \comp_Y(\tau)}}
            \right)} \\
        \intertext{by \autoref{notbad}}
        \leq& 1 - \expsub{\tau \sim Y}{
            \probsub{K_H \in \mathcal{K}_H}{(K_H, \tau) \notin \bad}} \\
        = & \probsub{\tau \sim Y, K_H \in \mathcal{K}_H}{(K_H, \tau) \in \bad} \\
        \leq & \probsub{\tau \sim Y, K_H \in \mathcal{K}_H}{(K_H, \tau) \in \badQ}
         + \probsub{\tau \sim Y, K_H \in \mathcal{K}_H}{(K_H, \tau) \in \badR} \\
         \intertext{by \autoref{badQ} and \autoref{badR}}
        \leq & (\epsilon + 2^{-n})\binom{q}{2} \\
    \end{align*}
\end{proof}

\subsection{Proof of primary claim}
\begin{proof}[Proof of \autoref{hbshadvantage}]\label{hbshproof}
    To prove this theorem we need a bound on $|\rho_V - \rho_Z|$
    where
    \begin{align*}
        \rho_V \defeq& \probsub{K_S \sample \mathcal{K}_S}
            {A^{\HBSH_{K_S}, \HBSH_{K_S}^{-1}}\Rightarrow 1} \\
        \rho_Z \defeq& \probsub{\bm{\pi} \sample \Perm^\mathcal{T}(\mathcal{M})}
            {A^{\bm{\pi},\bm{\pi}^{-1}}\Rightarrow 1} \\
    \end{align*}

    $|\rho_X - \rho_Y| \leq (\epsilon + 2^{-n})\binom{q}{2}$ by \autoref{xyadv}.
    Since we forbid pointless queries,
    $|\rho_Y - \rho_Z| \leq 2^{-n}\binom{q}{2}$ by Halevi and Rogaway's PRP-RND lemma
    (\cite{cmc}, appendix C, lemma 6).

    To bound $|\rho_V - \rho_X|$ we introduce
    a stepping stone. Let $\bar{\eta}_{F} \defeq \eta_{K_H, E_{K_E}, F}$ where
    $E$ is a block cipher and $K_E || K_H || \ldots = F(\lambda)$. Define
    \begin{align*}
        \rho_W \defeq& \probsub{F \sample (\mathcal{N} \rightarrow \bin^{l_S})}
            {A^{\bar{\eta}_{F}, \bar{\eta}_{F}^{-1}}\Rightarrow 1} \\
    \end{align*}

    Note that $\HBSH_{K_S} = \bar{\eta}_{S_{K_S}}$, so distinguishing
    $\rho_V$ and $\rho_W$ is just distinguishing the substitution of a PRF
    for a random function.
    Including the key schedule, the attacker distinguishing
    $\rho_V$ and $\rho_W$ makes at most $q + 1$ queries on the stream cipher
    or random function respectively, and uses at most $|K_E| + |K_H| + q(l_M - n)$ bits
    of the output; by a standard substitution argument
    $|\rho_V - \rho_W| \leq
    \advantage{\mathrm{prf}}{S}[(q + 1, |K_E| + |K_H| + q(l_M - n), t')]$
    where $t' = t + \bigO{q(l_T + l_M)}$.

    The differences between $\rho_W$ and $\rho_X$ are the use of a block cipher
    in place of a random permutation, and the use of $F(\lambda)$ to determine
    $K_E$ and $K_H$. Since $F$ is a random function and $F(\lambda)$ is used
    only here, this is equivalent to choosing them at random; again by a substitution
    argument we have that $|\rho_W - \rho_X| \leq \advantage{\pm \mathrm{prp}}{E}[(q, t')]$.

    \autoref{hbshadvantage} follows by summing these bounds:
    $|\rho_V - \rho_Z| \leq
    |\rho_V - \rho_W| + |\rho_W - \rho_X| + |\rho_X - \rho_Y| + |\rho_Y - \rho_Z|$.
\end{proof}

\subbib
\end{document}

\printbibliography
\appendix
% Copyright 2018 Google LLC
%
% Use of this source code is governed by an MIT-style
% license that can be found in the LICENSE file or at
% https://opensource.org/licenses/MIT.

%!TeX spellcheck = en-US

\documentclass[eprint.tex]{subfiles}
\begin{document}
\section{\texorpdfstring{$\epsilon$-$\Delta$U}{𝜖-∆U} functions for HBSH}\label{hashing}

Adiantum and HPolyC are identical except for the choice of $\epsilon$-$\Delta$U hash function
$H_{K_H}(T, L)$. In each case the value of $\epsilon$ depends on bounds on $\abs{T}$ and $\abs{L}$.
If queries to HBSH are bounded to a maximum tweak and plaintext/ciphertext length of
$\abs{T} \leq l_T$, $\abs{P}, \abs{C} \leq l_M$
then the bounds on queries to $H$ will be $\abs{T} \leq l_T$, $\abs{L} \leq l_L = l_M - n$.

\subsection{Poly1305}
For both Adiantum and HPolyC,
the output group for which the $\epsilon$-$\Delta$U property applies is
$\ZZ/2^{128}\ZZ$, so we define

\begin{align*}
    x \boxplus y &= \fromint_{128}(\intify(x) + \intify(y)) \\
    x \boxminus y &= \fromint_{128}(\intify(x) - \intify(y)) \\
\end{align*}

\cite{poly1305} uses polynomials over the finite field $\ZZ/(2^{130}-5)\ZZ$
to define a function we call
$\Polydjb: \bin^{128} \times \bin^* \rightarrow \bin^{128}$,
and proves in Theorem 3.3 that it is $\epsilon$-$\Delta$U: for any
$g \in \bin^{128}$ and any distinct messages $M, M'$ where $\abs{M}, \abs{M'} \leq l$,
$\probsub{K_H \sample{\bin^{128}}}{H_{K_H}(M') \boxminus H_{K_H}(M) = g} \leq 2^{-103}\ceil{l/128}$.
In that paper this function is used to build a MAC based on AES, while in
RFC 7539~\cite{RFC7539} it's used to build an AEAD mode based on ChaCha20.
Note that 22 bits of the 128-bit key are zeroed before use, so every key is equivalent to
$2^{22} - 1$ other keys and the effective keyspace is $2^{106}$.

\subsection{HPolyC hashing}
HPolyC is the HBSH construction that the first revision of this paper presented, which used
Poly1305 together with an injective encoding function.
It is simple, fast, and key agile.
\begin{align*}
\mathcal{T} ={}& \bigcup_{i=0}^{2^{32}-1}\bin^i \\
H_{K_H}(T, L) ={}& \Polydjb_{K_H}(\pad_{128}(\intify_{32}(\abs{T}) \Concat T) \Concat L) \\
\end{align*}

Thus if for all queries $\abs{T} \leq l_T$ and $\abs{L} \leq l_L$ then:

\begin{displaymath}
\epsilon = 2^{-103}(\ceil{(32 + l_T)/128} + \ceil{l_L/128})
\end{displaymath}\label{hpolycepsilon}

\subsection{NH}\label{nh}

We define a word size $w = 32$, a stride $s = 2$,
a number of rounds $r = 4$ and an input size $u = 8192$ such that $2sw$ divides $u$.

NH~\cite{umac1,umac2,rfc4418} is then defined over message
lengths divisible by $2sw = 128$
and takes a $u + 2sw(r -1) = 8576$-bit key, processing the message
in $u$-bit chunks to produce
an output of size $2rw\ceil{\abs{M}/u}$; we call this ratio $u/2rw = 32$ the ``compression ratio''.

\begin{algorithmic}[0]
    \Procedure{NH}{$K, M$}
    \State $h \gets \lambda$
    \While {$M \neq \lambda$}
        \State $l \gets \min{(\abs{M}, u)}$
        \For {$i \gets 0, 2sw, \ldots,  2sw(r-1)$}
            \State $p \gets 0$
            \For {$j \gets 0, 2sw, \ldots, l-2sw$}
                \For {$k \gets 0, w, \ldots, w(s-1)$}
                    \State $a_0 \gets \intify(K[i+j+k;w])$
                    \State $a_1 \gets \intify(K[i+j+k+sw;w])$
                    \State $b_0 \gets \intify(M[j+k;w])$
                    \State $b_1 \gets \intify(M[j+k+sw;w])$
                    \State $p \gets p + ((a_0 + b_0) \bmod 2^w)((a_1 + b_1) \bmod 2^w)$
                \EndFor
            \EndFor
            \State $h \gets h \Concat \fromint_{2w}(p)$
        \EndFor
        \State $M \gets M[l;\abs{M} - l]$
    \EndWhile
    \State \textbf{return} $h$
    \EndProcedure
\end{algorithmic}

This is the largest $w$ where common vector instruction sets (NEON on ARM; SSE2
and AVX2 on x86) natively support the needed $\bin^w \times \bin^w \rightarrow
\bin^{2w}$ multiply operation.  The stride $s=2$ improves vectorization on
ARM32 NEON; larger strides were slower or no faster on every platform we tested
on. We choose $r=4$ since we want $\epsilon = 2^{-rw} \leq 2^{-103}$ to match
HPolyC, and a large $u$ for a high compression ratio which reduces the work for
the next hashing stage.

NH's speed comes with several inconvenient properties:
\begin{itemize}
    \item \cite{umac2} shows that this function is $\epsilon$-almost-$\Delta$-universal, but this
        holds only over equal-length inputs
    \item $\epsilon = 2^{-rw}$, but the smallest nonempty output is $2rw$ bits, twice as large
        as necessary for this $\epsilon$ value
    \item The output size varies with the input size.
\end{itemize}
A second hashing stage is used to handle these issues.

\subsection{Adiantum hashing}

For Adiantum we use NH followed by Poly1305 to hash the message.
Our theorems assume $\mathcal{T}$ is finite,
so we somewhat arbitrarily set
$\mathcal{T} = \mathcal{L} = \bigcup_{i=0}^{l_S}\bin^i$.
To avoid encoding and padding issues, we hash the message length and tweak with
a separate Poly1305 key.
In all this takes a $128 + 128 + 8576 = 8832$-bit key.

\begin{algorithmic}[0]
    \Procedure{H}{$K_H,T,L$}
    \State $K_T \gets K_H[0;128]$
    \State $K_L \gets K_H[128;128]$
    \State $K_N \gets K_H[256;8576]$
    \State $H_T \gets \Polydjb_{K_T}(\fromint_{128}(\abs{L}) \Concat T)$
    \State $H_L \gets \Polydjb_{K_L}(\NH_{K_N}(\pad_{128}(L)))$
    \State \textbf{return} $H_T \boxplus H_L$
    \EndProcedure
\end{algorithmic}

For distinct pairs $(T,L) \neq (T', L')$, we have that if $\abs{L} \neq \abs{L'}$ or $T \neq T'$,
then the $128 + \abs{T}$-bit input to Poly1305 with key $K_T$ will differ.
Otherwise $\abs{L} = \abs{L'}$ but $L \neq L'$;
per \cite{umac2} the probability NH will compress these to the same value is at most
$2^{-128}$. If they do not collide, the $256\ceil{\abs{L}/8192}$-bit input to Poly1305 with key $K_L$
will differ. Since the sum of two $\epsilon$-$\Delta$U functions with independent keys is also
$\epsilon$-$\Delta$U, if for all queries $\abs{T} \leq l_T$ and $\abs{L} \leq l_L$ then
this composition is  $\epsilon$-$\Delta$U, with:

\begin{align*}
\epsilon &= 2^{-128} + 2^{-103}\ceil{\max(128 + l_T, 256\ceil{l_L/8192})/128}  \\
&= 2^{-128} + 2^{-103}\max(1 + \ceil{l_T/128}, 2\ceil{l_L/8192})
\end{align*}\label{adiantumepsilon}

\subsection{Usage limits}
If we limit our Adiantum adversary to at most $q$ queries each of which uses a tweak of length at
at most $l_T$ and a plaintext/ciphertext of length at most $l_M$, then by \autoref{hbshadvantage}
their distinguishing advantage is therefore at most:

\begin{align*}
{}&( 3(2^{-128}) + 2^{-103}\max(1 + \ceil{l_T/128}, 2\ceil{(l_M - 128)/8192}))\binom{q}{2} \\
+{}& \advantage{\mathrm{sc}}{S_{K_S}}[(q + 1, 256 + 8832 + q(l_M - 128), t')] \\
+{}& \advantage{\pm \mathrm{prp}}{E_{K_E}}[(q, t')] \\
\end{align*}

Assuming that the block and stream ciphers are strong, the advantage is dominated by the
term for internal collisions: $2^{-103}\max(1 + \ceil{l_T/128}, 2\ceil{(l_M -
128)/8192})\binom{q}{2}$. How many messages can be safely encrypted with the
mode will therefore vary with message and tweak length. For example, if Adiantum
is used to encrypt 4KiB sectors with 32 byte tweaks, then $\Polydjb_{K_L}$
processes 8 blocks, and the above is approximately $2^{-101}q^2$. With these
message and tweak lengths we would recommend encrypting no more than $2^{55}$
bytes with a single key. Generating the ciphertext to mount such an attack could
be very time-consuming, and this is work that can only be done on the device
that has the key; extrapolating from performance figures in
\autoref{performance}:

\vspace{0.3cm}
\begin{tabular}{llll}
    Bytes of ciphertext & Advantage & Time on device (single-threaded) \\
    \hline
    512GiB & $2^{-47}$ & 80 minutes  \\
    $2^{55}$ & $2^{-15}$ & 11 years \\
    $2^{59}$ & 0.8\% & 175 years &
\end{tabular}
\vspace{0.3cm}

\subbib
\end{document}


\vspace*{\fill}
Version: \texttt{\input{work/git.tex}}
\end{document}
