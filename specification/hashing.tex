% Copyright 2018 Google LLC
%
% Use of this source code is governed by an MIT-style
% license that can be found in the LICENSE file or at
% https://opensource.org/licenses/MIT.

%!TeX spellcheck = en-US

\documentclass[eprint.tex]{subfiles}
\begin{document}
\section{\texorpdfstring{$\epsilon$A$\Delta$U}{𝜖A∆U} functions for HBSH}\label{hashing}

Adiantum and HPolyC are identical except for the choice of $\epsilon$A$\Delta$U hash function
$H_{K_H}(T, L)$. In each case the value of $\epsilon$ depends on bounds on $|T|$ and $|L|$.
If queries to HBSH are bounded to a maximum tweak and plaintext/ciphertext length of
$|T| \leq l_T$, $|P|, |C| \leq l_M$
then the bounds on queries to $H$ will be $|T| \leq l_T$, $|L| \leq l_L = l_M - n$.

\subsection{Notation}

$\intify: \bin^{*} \rightarrow \ZZ$ is the
standard little-endian map
(ie $\intify(\lambda) = 0$, $\intify(0 || X) = 2\intify(X)$, $\intify(1 || X) = 1 + 2\intify(X)$),
and $X = \fromint_l(y)$ is the unique
$l$-bit sequence such that $\intify(X) \equiv y \pmod{2^l}$.

For both Adiantum and HPolyC,
the output group for which the $\epsilon$A$\Delta$U property applies is
$\ZZ/2^{128}\ZZ$, so we define

\begin{align*}
    x \boxplus y &= \fromint_{128}(\intify(x) + \intify(y)) \\
    x \boxminus y &= \fromint_{128}(\intify(x) - \intify(y)) \\
\end{align*}

\subsection{Poly1305}

\cite{poly1305} uses polynomials over the finite field $\ZZ/(2^{130}-5)\ZZ$
to define a function we call
$\Polydjb: \bin^{128} \times \bin^* \rightarrow \bin^{128}$,
and proves in Theorem 3.3 that it is $\epsilon$A$\Delta$U: for any
$g \in \bin^{128}$ and any distinct messages $M, M'$ where $|M|, |M'| \leq l$,
$\probsub{K_H \sample{\bin^{128}}}{H_{K_H}(M') \boxminus H_{K_H}(M) = g} \leq 2^{-103}\ceil{l/128}$.
In that paper this function is used to build a MAC based on AES, while in
RFC 7539~\cite{RFC7539} it's used to build an AEAD mode based on ChaCha20.
Note that 22 bits of the 128-bit key are zeroed before use, so every key is equivalent to
$2^{22}$ keys and the effective keyspace is $2^{106}$.

Many Poly1305 libraries
take parameters $K_H || g, M$ and return $g \boxplus \Polydjb_{K_H}(M)$; where subtraction
is needed we suggest using bitwise inversion and the identity
$g \boxminus g' = \neg ((\neg g) \boxplus g')$.

\subsection{HPolyC hashing}
HPolyC is the HBSH construction that the first revision of this paper presented, which used
Poly1305 together with an injective encoding function.
It is simple, fast, and key agile. We require that $|T| < 2^{32}$ and define
\begin{align*}
H_{K_H}(T, L) = \Polydjb_{K_H}(\pad_{128}(\intify_{32}(|T|) || T) || L)
\end{align*}

Thus if for all queries $|T| \leq l_T$ and $|L| \leq l_L$ then:

\begin{displaymath}
\epsilon = 2^{-103}(\ceil{(32 + l_T)/128} + \ceil{l_L/128})
\end{displaymath}\label{hpolycepsilon}

\subsection{NH}\label{nh}

We define a word size $w = 32$, a stride $s = 2$,
a number of rounds $r = 4$ and an input size $u = 8192$ such that $2sw$ divides $u$.

NH~\cite{umac1,umac2,rfc4418} is then defined over message
lengths divisible by $2sw = 128$
and takes a $u + 2sw(r -1) = 8576$-bit key, processing the message
in $u$-bit chunks to produce
an output of size $2rw\ceil{|M|/u}$; we call this ratio $u/2rw = 32$ the ``compression ratio''.

\begin{algorithmic}[0]
    \Procedure{NH}{$K, M$}
    \State $h \gets \lambda$
    \While {$M \neq \lambda$}
        \State $l \gets \min{(|M|, u)}$
        \For {$i \gets 0, 2sw, \ldots,  2sw(r-1)$}
            \State $p \gets 0$
            \For {$j \gets 0, 2sw, \ldots, l-2sw$}
                \For {$k \gets 0, w, \ldots, w(s-1)$}
                    \State $a_0 \gets \intify(K[i+j+k;w])$
                    \State $a_1 \gets \intify(K[i+j+k+sw;w])$
                    \State $b_0 \gets \intify(M[j+k;w])$
                    \State $b_1 \gets \intify(M[j+k+sw;w])$
                    \State $p \gets p + ((a_0 + b_0) \bmod 2^w)((a_1 + b_1) \bmod 2^w)$
                \EndFor
            \EndFor
            \State $h \gets h || \fromint_{2w}(p)$
        \EndFor
        \State $M \gets M[l;|M| - l]$
    \EndWhile
    \State \textbf{return} $h$
    \EndProcedure
\end{algorithmic}

This is the largest $w$ where common vector instruction sets (NEON on ARM; SSE2
and AVX2 on x86) natively support the needed $\bin^w \times \bin^w \rightarrow
\bin^{2w}$ multiply operation.  The stride $s=2$ improves vectorization on
ARM32 NEON; larger strides were slower or no faster on every platform we tested
on. We choose $r=4$ since we want $\epsilon = 2^{-rw} \leq 2^{-103}$ to match
HPolyC, and a large $u$ for a high compression ratio which reduces the work for
the next hashing stage.

NH's speed comes with several inconvenient properties:
\begin{itemize}
    \item \cite{umac2} shows that this function is $\epsilon$-almost-$\Delta$-universal, but this
        holds only over equal-length inputs
    \item $\epsilon = 2^{-rw}$, but the smallest nonempty output is $2rw$ bits, twice as large
        as necessary for this $\epsilon$ value
    \item The output size varies with the input size.
\end{itemize}
A second hashing stage is used to handle these issues.

\subsection{Adiantum hashing}

For Adiantum we use NH followed by Poly1305 to hash the message.
To avoid encoding and padding issues, we hash the message length and tweak with
a separate Poly1305 key.
In all this takes a $128 + 128 + 8576 = 8832$-bit key.

\begin{algorithmic}[0]
    \Procedure{H}{$K_H,T,L$}
    \State $K_T \gets K_H[0;128]$
    \State $K_L \gets K_H[128;128]$
    \State $K_N \gets K_H[256;8576]$
    \State $H_T \gets \Polydjb_{K_T}(\fromint_{128}(|L|) || T)$
    \State $H_L \gets \Polydjb_{K_L}(\NH_{K_N}(\pad_{128}(L)))$
    \State \textbf{return} $H_T \boxplus H_L$
    \EndProcedure
\end{algorithmic}

For distinct pairs $(T,L) \neq (T', L')$, we have that if $|L| \neq |L'|$ or $T \neq T'$,
then the $128 + |T|$-bit input to Poly1305 with key $K_T$ will differ.
Otherwise $|L| = |L'|$ but $L \neq L'$;
per \cite{umac2} the probability NH will compress these to the same value is at most
$2^{-128}$. If they do not collide, the $256\ceil{|L|/8192}$-bit input to Poly1305 with key $K_L$
will differ. Since the sum of two $\epsilon$A$\Delta$U functions with independent keys is also
$\epsilon$A$\Delta$U, if for all queries $|T| \leq l_T$ and $|L| \leq l_L$ then
this composition is  $\epsilon$A$\Delta$U, with:

\begin{align*}
\epsilon &= 2^{-128} + 2^{-103}\ceil{\max(128 + l_T, 256\ceil{l_L/8192})/128}  \\
&= 2^{-128} + 2^{-103}\max(1 + \ceil{l_T/128}, 2\ceil{l_L/8192})
\end{align*}\label{adiantumepsilon}

If we limit our Adiantum attacker to at most $q$ queries each of which uses a tweak of length at
at most $l_T$ and a plaintext/ciphertext of length at most $l_M$, then by \autoref{hbshadvantage}
their distinguishing advantage is therefore at most:

\begin{align*}
{}&( 3(2^{-128}) + 2^{-103}\max(1 + \ceil{l_T/128}, 2\ceil{(l_M - 128)/8192}))\binom{q}{2} \\
+{}& \advantage{\pm \mathrm{prp}}{E_{K_E}}[(q, t')] \\
+{}& \advantage{\mathrm{prf}}{S_{K_S}}[(8832 + q(l_M - 128), t')] \\
\end{align*}

Assuming that the block and stream ciphers are strong, this is dominated by the
term for internal collisions: $2^{-103}\max(1 + \ceil{l_T/128}, 2\ceil{(l_M -
128)/8192})\binom{q}{2}$. How many messages can be safely encrypted with the
mode will therefore vary with message and tweak length. For example, if Adiantum
is used to encrypt 4KiB sectors with 32 byte tweaks, then $\Polydjb_{K_L}$
processes 8 blocks, and the above is approximately $2^{-101}q^2$. With these
message and tweak lengths we would recommend encrypting no more than $2^{55}$
bytes with a single key. Generating the ciphertext to mount such an attack could
be very time-consuming, and this is work that can only be done on the device
that has the key; extrapolating from performance figures in
\autoref{performance}:

\vspace{0.3cm}
\begin{tabular}{llll}
    Bytes of ciphertext & Advantage & Time on device (single-threaded) \\
    \hline
    512GiB & $2^{-47}$ & 80 minutes  \\
    $2^{55}$ & $2^{-15}$ & 11 years \\
    $2^{59}$ & 0.8\% & 175 years &
\end{tabular}
\vspace{0.3cm}

\subbib
\end{document}
