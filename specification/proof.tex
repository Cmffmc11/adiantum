% Copyright 2018 Google LLC
%
% Use of this source code is governed by an MIT-style
% license that can be found in the LICENSE file or at
% https://opensource.org/licenses/MIT.

%!TeX spellcheck = en-US

\documentclass[eprint.tex]{subfiles}
\begin{document}
\section{Security of HBSH}
Assuming the security of the underlying block and stream ciphers,
we show here that HBSH has an
advantage bound that grows quadratically with the number of queries.
\subsection{Definition of HBSH}\label{funcdef}
We provide here an
equivalent definition of HBSH in functional form.
Where a parameter is given as $L \Concat R$ we have that
$L \in \mathcal{L}$, $R \in \mathcal{R}$, $L \Concat R \in \mathcal{M}$.

\begin{align*}
    \xi :{}& \mathcal{K}_H \times \mathcal{T} \times \mathcal{M} \rightarrow \mathcal{R} \\
    \xi_{K_H}(T, L \Concat R) \defeq{}& R \boxplus H_{K_H}(T, L) \\
    %
    \allowdisplaybreaks[4] \vphantom{|} \\ \allowdisplaybreaks[0]
    %
    \phi :{}& \mathcal{K}_H \times \mathcal{T} \times \mathcal{M} \rightarrow \mathcal{M} \\
    \phi_{K_H, T}(L \Concat R) \defeq{}& L \Concat \xi_{K_H}(T, L \Concat R) \\
    ={}& L \Concat (R \boxplus H_{K_H}(T, L)) \\
    \phi^{-1}_{K_H, T}(L \Concat R) ={}& L \Concat (R \boxminus H_{K_H}(T, L)) \\
    %
    \allowdisplaybreaks[4] \vphantom{|} \\ \allowdisplaybreaks[0]
    %
    \theta :{}& \Perm(\mathcal{R}) \times (\mathcal{N} \rightarrow \bin^{l_S}) \times \mathcal{M} \rightarrow \mathcal{M} \\
    \theta_{\pi, F}(L \Concat R) \defeq{}& (L \oplus F(\pi(R))[0;\abs{L}]) \Concat \pi(R) \\
    \theta_{\pi, F}^{-1}(L \Concat R) ={}& (L \oplus F(R)[0;\abs{L}]) \Concat \pi^{-1}(R) \\
    %
    \allowdisplaybreaks[4] \vphantom{|} \\ \allowdisplaybreaks[0]
    %
    \eta :{}& \mathcal{K}_H \times \Perm(\mathcal{R}) \times (\mathcal{N} \rightarrow \bin^{l_S}) \times \mathcal{T} \times \mathcal{M} \rightarrow \mathcal{M} \\
    \eta_{K_H, \pi, F, T} \defeq{}& \phi_{K_H,T}^{-1} \circ \theta_{\pi, F} \circ \phi_{K_H,T} \\
    %
    \allowdisplaybreaks[4] \vphantom{|} \\ \allowdisplaybreaks[0]
    %
    \bar{\eta} :{}& (\mathcal{N} \rightarrow \bin^{l_S}) \times \mathcal{T} \times \mathcal{M} \rightarrow \mathcal{M} \\
    \bar{\eta}_{F} \defeq{}& \eta_{K_H, E_{K_E}, F} \quad \textrm{where} \ K_E \Concat K_H \Concat \ldots = F(\lambda) \\
    %
    \allowdisplaybreaks[4] \vphantom{|} \\ \allowdisplaybreaks[0]
    %
    \HBSH :{}& \mathcal{K}_S \times \mathcal{T} \times \mathcal{M} \rightarrow \mathcal{M} \\
    \HBSH_{K_S} \defeq{}& \bar{\eta}_{S_{K_S}} \\
\end{align*}

\subsection{Security definitions}
\parintro{Hash function}
The hash function $H$
must be $\epsilon$-almost-$\Delta$-universal\label{eadudef}
($\epsilon$-$\Delta$U)
for some $\epsilon$~\cite{eadu}:
for any $g \in \mathcal{R}$ and
any two distinct messages $(T, L) \neq (T', L')$:
%
\begin{displaymath}
\probsub{K \sample \mathcal{K}_H}{H_{K}(T, L) \boxminus H_{K}(T', L') = g} \leq \epsilon
\end{displaymath}
%
Given bounds on the lengths of $T$ and $L$, the value of $\epsilon$ for the
hash function used in HPolyC is given in \autoref{hpolycepsilon},
and for Adiantum in \autoref{adiantumepsilon}.

\parintro{Block cipher}
The block cipher
$E$
must be a super-pseudorandom permutation~\cite{concsym}:
%
\begin{align*}
    \advantage{\pm \mathrm{prp}}{E}[(A)] \defeq
    {}&\left\lvert\probsub{K \sample \mathcal{K}_E}{A^{E_K,E_K^{-1}}\Rightarrow 1}\right.
    \\
    {}&\left. - \probsub{\pi \sample \Perm(\mathcal{R})}{A^{\pi,\pi^{-1}}\Rightarrow 1}\right\rvert
    \\
    \advantage{\pm \mathrm{prp}}{E}[(q, t)] \defeq
    {}&\max_{A \in \mathcal{A}(q, t)} \advantage{\pm \mathrm{prp}}{E}[(A)]
\end{align*}
%
where $A$ is an adversary,
$\Perm(S)$ denotes the set of all permutations on a set $S$,
and
$\mathcal{A}(q, t)$
is the set of all adversaries that make at most $q$ queries and take at most $t$ time.

\parintro{Stream cipher}
Our definition is related to the definition of a PRF in \cite{concsym}, but
because we model the stream cipher $S$ as a pseudorandom function with a very long output,
we bound the adversary not only in how many queries they make,
but in how many bits they read in total.
Thus, a query consists of a pair
$(N, l_q) \in \mathcal{N} \times \NN$ where $0 < l_q \leq l_S$,
and the response is $S_K(N)[0;l_q]$ or $F(N)[0;l_q]$;
we cap the sum of $l_q$ values across queries.
%
\begin{align*}
    \advantage{\mathrm{sc}}{S}[(A)] \defeq
    {}&\left\lvert \probsub{K \sample \mathcal{K}_S}{A^{S_{K}(.)[0;.]}\Rightarrow 1}\right.
    \\
    {}&\left. - \probsub{F \sample (\mathcal{N} \rightarrow \bin^{l_S})}
    {A^{F(.)[0;.]}\Rightarrow 1} \right\rvert
    \\
    \advantage{\mathrm{sc}}{S}[(q, l, t)]
    \defeq {}&\max_{A \in \mathcal{A}(q, l, t)} \advantage{\mathrm{sc}}{S}[(A)]
\end{align*}
%
where $A$ is an adversary,
$\mathcal{N} \rightarrow \bin^{l_S}$ denotes the set of all
functions from $\mathcal{N}$ to $\bin^{l_S}$,
and
$\mathcal{A}(q, l, t)$
is the set of all adversaries that make at most $q$ queries such that
$\sum l_q \leq l$, and which take at most $t$ time.

\parintro{Tweakable SPRP}
Let $\LP^\mathcal{T}(\mathcal{M})$ denote the set of all
tweakable length-preserving functions
$\bm{f} : \mathcal{T} \times \mathcal{M} \rightarrow \mathcal{M}$
such that for all $T, M \in \mathcal{T} \times \mathcal{M}$,
$\abs{\bm{f}(T, M)} = \abs{M}$. Let $\Perm^\mathcal{T}(\mathcal{M})$ denote
the set of $\bm{\pi} \in \LP^\mathcal{T}(\mathcal{M})$ such that
for all $T \in \mathcal{T}$, $\bm{\pi}_{T}$ is a bijection.
In an abuse of notation
we use $\bm{\pi}^{-1}$ to refer to the function
such that $\bm{\pi}^{-1}(T, \bm{\pi}(T, M)) = M$ ie $(\bm{\pi}^{-1})_T = (\bm{\pi}_T)^{-1}$.

Per~\cite{cmc}, for a tweakable, variable-input-length, super-pseudorandom permutation
$\bm{E} : \mathcal{K} \times \mathcal{T} \times \mathcal{M} \rightarrow \mathcal{M}$
the distinguishing advantage of an adversary $A$ is:
%
\begin{align*}
    \advantage{\pm \widetilde{\mathrm{prp}}}{\bm{E}}[(A)] \defeq
    {}&\left\lvert\probsub{K \sample \mathcal{K}}{A^{\bm{E}_K,\bm{E}_K^{-1}}\Rightarrow 1}\right.
    \\
    {}&\left. - \probsub{\bm{\pi} \sample \Perm^\mathcal{T}(\mathcal{M})}
        {A^{\bm{\pi},\bm{\pi}^{-1}}\Rightarrow 1}\right\rvert
    \\
    \intertext{and}
    \advantage{\pm \widetilde{\mathrm{prp}}}{\bm{E}}[(q, l_T, l_M, t)]
    \defeq {}&
    \max_{A \in \mathcal{A}(q, l_T, l_M, t)} \advantage{\pm \widetilde{\mathrm{prp}}}{\bm{E}}[(A)]
\end{align*}
%
\begin{displaymath}
\end{displaymath}
where $\mathcal{A}(q, l_T, l_M, t)$
is the set of all adversaries that
make at most $q$ queries,
with tweak of length at most $l_T$
and message of length at most $l_M$,
and take at most $t$ time.

\subsection{Primary claim}
\begin{theorem}\label{hbshadvantage}
    Where HBSH mode is instantiated with hash function $H$, block cipher $E$ and stream cipher $S$,
    and where $H$ is $\epsilon$-almost-$\Delta$-universal for inputs such that
    $\abs{T} \leq l_T$, $\abs{L} \leq l_M - n$, then:
    %
    \begin{align*}
        \advantage{\pm \widetilde{\mathrm{prp}}}{\HBSH}[(q, l_T, l_M, t)]
        \leq {}&(\epsilon + 2(2^{-n}))\binom{q}{2} \\
        {}&+ \advantage{\mathrm{sc}}{S}[(q + 1, \abs{K_E} + \abs{K_H} + q(l_M - n), t')] \\
        {}&+ \advantage{\pm \mathrm{prp}}{E}[(q, t')]\\
    \end{align*}
    %
    where $t' = t + \bigO{q(l_T + l_M)}$.
\end{theorem}

This is proven in what follows. First we use the H-coefficient technique to establish
\autoref{xyadv}, a closely related bound; then in \autoref{hbshproof} we bridge the
gap between this bound and the desired bound.

\subsection{H-coefficient technique}
The H-coefficient technique was introduced by Patarin in 1991~\cite{ppdes,hco}.
In what follows we rely on the highly recommended exposition of
\cite{hco2} Section 3,
``The H-coefficient Technique in a Nutshell'', though we vary slightly by introducing
a new symbol $\Upsilon$ so we can distinguish between what is sampled and the adversary oracles.

We wish to bound the adversary's ability to distinguish between
two ``worlds'', world X (the ``real world'') and world Y (the ``ideal world'').
Associated with world X we have
\begin{itemize}
    \item $\Omega_X$: a set of instances we sample fairly from. We write
    $\probsub{\Omega_X}{}$ as shorthand for $\probsub{\omega \sample \Omega_X}{}$.
    \item $\Upsilon_X$: a map from an instance $\omega \in \Omega_X$ to a tuple of
    deterministic oracles we can present to the adversary.
    \item $\rho_X \defeq \probsub{\Omega_X}{A^{\Upsilon_X(\omega)} \Rightarrow 1}$\label{rhox}
    where the adversary $A$ is clear from context.
    As the adversary interacts with the oracles, a transcript $\tau$
    of queries and responses is generated.
    \item $X$: a random variable representing a transcript for $A^{\Upsilon_X(\omega)}$
    given $\omega \sample \Omega_X$; we write $\tau \sample X$
    to indicate that $\tau$ is sampled from this distribution.
    \item $\comp_X$:  We write $\omega \in \comp_X(\tau)$
    if a transcript $\tau$ is ``compatible'' with an instance $\omega \in \Omega_X$,
    ie if given an adversary $A$ that makes those queries, the oracles $\Upsilon_X(\omega)$
    make those responses and thus $A^{\Upsilon_X(\omega)}$ produces that transcript.
\end{itemize}
We have the same for world Y throughout.

We assume a deterministic adversary. This is without loss of
generality; if we assume a distribution of adversaries $A \sample \mathcal{A}$
then an advantage bound on each of the deterministic adversaries $A$ bounds the advantage
of the ensemble $\mathcal{A}$.

Once $\omega$ is sampled, the oracles $\Upsilon_X(\omega)$ are then deterministic;
the transcript produced by $A^{\Upsilon_X(\omega)}$ is thus the unique transcript
compatible both with adversary $A$ and instance $\omega$. Where a transcript
is not compatible with $A$, $\prob{X = \tau} = \prob{Y = \tau} = 0$. If either
of these is not zero, the transcript is compatible with $A$, and
$\prob{X = \tau} = \probsub{\Omega_X}{\omega \in \comp_X(\tau)}$ and similarly for Y.

The adversary always returns the same result for the same transcript, so
its advantage is maximized if it returns 1 exactly when $\prob{Y = \tau} > \prob{X = \tau}$.
Therefore:
\begin{align*}
    \advantage{\textrm{Y}}{\textrm{X}}[(A)] ={}&
    \abs{\rho_X - \rho_Y} \\
    \leq {}& \sum_{\tau: \prob{Y = \tau} > \prob{X = \tau}}
    \left(\prob{Y = \tau} - \prob{X = \tau}\right) \\
    = {}& \sum_{\tau: \prob{Y = \tau} > \prob{X = \tau}}
    \prob{Y = \tau}\left(1 - \frac{\prob{X = \tau}}{\prob{Y = \tau}}\right) \\
    = {}& \sum_{\tau: \prob{Y = \tau} > 0}
    \prob{Y = \tau}\left(1 - \minone{\frac{\prob{X = \tau}}{\prob{Y = \tau}}}\right) \\
    = {}& \expsub{\tau \sample Y}{1 - \minone{
           \frac{\prob{X = \tau}}{\prob{Y = \tau}}
        }} \\
    = {}& 1 - \expsub{\tau \sample Y}{\minone{
           \frac{\probsub{\Omega_X}{\omega \in \comp_X(\tau)}}
           {\probsub{\Omega_Y}{\omega \in \comp_Y(\tau)}}
        }}
\end{align*}
where $\expect{}$ is the expected value.
With this rearrangement, the only probability distribution we sum over is that
of $Y$, which can be more convenient to work with.

\subsection{Preliminaries}
\parintro{World X}
World X is
an idealized form of HBSH which uses a random function and permutation:
$\Omega_X \defeq \mathcal{K}_H \times \Perm(\mathcal{R}) \times (\mathcal{N} \rightarrow \bin^{l_S})$,
and given $(K_H, \pi, F) \in \Omega_X$,
$\Upsilon_X(K_H, \pi, F) \defeq (\eta_{K_H, \pi, F}, \eta_{K_H, \pi, F}^{-1})$.

\parintro{World Y}
World Y samples fairly from all possible pairs of tweakable length-preserving functions:
$\Omega_Y \defeq \LP^\mathcal{T}(\mathcal{M}) \times \LP^\mathcal{T}(\mathcal{M})$,
so given $(\mathcal{E}, \mathcal{D}) \in \Omega_Y$,
$\Upsilon_Y(\mathcal{E}, \mathcal{D}) \defeq (\mathcal{E}, \mathcal{D})$.

\parintro{Transcript}
Our transcript $\tau$ is a sequence of tuples
$(d^i, T^i, P^i, C^i)$
in
$\{+, -\} \times \mathcal{T} \times \mathcal{M} \times \mathcal{M}$
for $i \in [0 \ldots q-1]$.
For the $i$th sequential query
$d^i$ is the direction of the query:
if $d^i = +$ then a plaintext query $T^i, P^i$ is made and the result is $C^i$,
while if $d^i = -$ then a ciphertext query $T^i, C^i$ is made and the result is $P^i$.

\parintro{Pointless queries}
We consider adversaries contained in $\mathcal{A}(q, l_T, l_M, t)$ for some value of
the bounds $q$, $l_T$, $l_M$, $t$.
Without loss of generality, we consider only adversaries who do not make ``pointless''
queries as defined in \cite{cmc}. Thus for $i < j$, if $d^j = +$ then
$(T^i, P^i) \neq (T^j, P^j)$, and similarly if $d^j = -$ then
$(T^i, C^i) \neq (T^j, C^j)$.

\parintro{Bad events}
We define various classes of ``bad event'':

\begin{itemize}
    \item $(K_H, \tau) \in \badQ$ if there exists $i < j$ such that either
    \begin{itemize}
        \item $d^j = +$ and $\xi_{K_H}(T^i, P^i) = \xi_{K_H}(T^j, P^j)$, or
        \item $d^j = -$ and $\xi_{K_H}(T^i, C^i) = \xi_{K_H}(T^j, C^j)$.
    \end{itemize}
    \item $(K_H, \tau) \in \badR$ if there exists $i < j$ such that either
    \begin{itemize}
        \item $d^j = +$ and $\xi_{K_H}(T^i, C^i) = \xi_{K_H}(T^j, C^j)$, or
        \item $d^j = -$ and $\xi_{K_H}(T^i, P^i) = \xi_{K_H}(T^j, P^j)$.
    \end{itemize}
\end{itemize}

Finally we define the disjunction
$\bad \defeq \badQ \cup \badR$.

\subsection{Lemmas}
\begin{lemma} \label{badQ}
    For any $\tau$ such that $\prob{Y = \tau} > 0$,
    \begin{displaymath}
        \probsub{K_H \sample \mathcal{K}_H}{(K_H, \tau) \in \badQ}
        \leq \epsilon \binom{q}{2}
    \end{displaymath}
\end{lemma}

\begin{proof}
Assume $d^j = +$ for some pair $i, j$, and let $L^i \Concat R^i = P^i$ and similarly for $P^j$.
From $\prob{Y = \tau} > 0$ we know that $\abs{T^i}, \abs{T^j} \leq l_T$ and $\abs{P^i}, \abs{P^j} \leq l_M$,
and therefore that $\abs{L^i}, \abs{L^j} \leq l_M - n$.
Because pointless queries are forbidden we also know that $(T^i, P^i) \neq (T^j, P^j)$.
%
\begin{align*}
    {}&\xi_{K_H}(T^i, L^i \Concat R^i) = \xi_{K_H}(T^j, L^j \Concat R^j) \\
    \Leftrightarrow{}& R^i \boxplus H_{K_H}(T^i, L^i) = R^j \boxplus H_{K_H}(T^j, L^j) \\
    \Leftrightarrow{}& H_{K_H}(T^i, L^i) \boxminus H_{K_H}(T^j, L^j) = R^j \boxminus R^i \\
\end{align*}
%
If $(T^i, L^i) = (T^j, L^j)$ then $R^i \neq R^j$ and equality cannot occur.
Otherwise by the $\epsilon$-$\Delta$U property of $H$ this occurs with probability
at most $\epsilon$ (where $\epsilon$ depends on the bounds on
the parameters $l_T$, $l_M -n$).

Where $d^j = -$, a similar argument applies for $C^i$, $C^j$.
For an upper bound, we sum across all $\binom{q}{2}$ pairs $i, j$.
\end{proof}

\begin{lemma} \label{badR}
    For any $K_H \sample \mathcal{K}_H$,
    \begin{displaymath}
        \probsub{\tau \sample Y}{(K_H, \tau) \in \badR}
        \leq 2^{-n} \binom{q}{2}
    \end{displaymath}
\end{lemma}

\begin{proof}
    Assume $d^j = +$ for some pair $i, j$, and let $L^i \Concat R^i = C^i$ and similarly for $C^j$.
    Because pointless queries are forbidden, in world Y,
    conditioning on all prior queries and responses,
    all possible values of $C^j$ such that $\abs{C^j} = \abs{P^j}$ will be equally likely.
    In particular, even after conditioning on $L^j$,
    all values of $R^j$ are equally likely. Therefore
    $\prob{R^j = R^i \boxplus H_{K_H}(T^i, L^i) \boxminus H_{K_H}(T^j, L^j)} = 2^{-n}$.

    Where $d^j = -$, a similar argument applies for $P^i$, $P^j$.
    For an upper bound, we sum across all $\binom{q}{2}$ pairs $i, j$.
\end{proof}

\begin{lemma} \label{notbad}
    For any $K_H \in \mathcal{K}_H$ and transcript $\tau$ such that $\prob{Y = \tau} > 0$ and
    $(K_H, \tau) \notin \bad$,
    \begin{displaymath}
        \condprobsub{\Omega_X}{\omega \in \comp_X(\tau)}{\omega = (K_H, ., .)}
        \geq
        \probsub{\Omega_Y}{\omega \in \comp_Y(\tau)}
    \end{displaymath}
\end{lemma}

\begin{proof}
    In world Y, for any transcript such that $\prob{Y = \tau} > 0$,
    since all queries are distinct, the responses are independent fair random draws of
    binary strings of the appropriate length, and therefore
    $\probsub{\Omega_Y}{\omega \in \comp_Y(\tau)} = \prod_i 2^{-\abs{P^i}}$.

    For world X,
    let $P_L^i \Concat P_R^i = P^i$, $P_M^i = \xi_{K_H, T^i}(P^i)$ and similarly for $C^i$.
    Since $(K_H, \tau) \notin \bad$ we have that $P_M^i \neq P_M^j$
    and $C_M^i \neq C_M^j$ for all $i \neq j$.
    $(K_H, \pi, F) \in \comp_X(\tau)$ exactly if, for each $i$:
    \begin{align*}
        {}& \eta_{K_H, \pi, F, T^i}(P^i) = C^i\\
        \Leftrightarrow {}& \phi_{K_H,T^i}^{-1}(\theta_{\pi, F}(\phi_{K_H,T^i}(P^i))) = C^i\\
        \Leftrightarrow {}& \theta_{\pi, F}(P_L^i \Concat P_M^i) = C_L^i \Concat C_M^i \\
        \Leftrightarrow {}& P_L^i \oplus F(C_M^i)[0;\abs{P_L^i}] = C_L^i \wedge \pi(P_M^i) = C_M^i \\
        \Leftrightarrow {}& F(C_M^i)[0;\abs{P^i} - n] = P_L^i \oplus C_L^i \wedge \pi(P_M^i) = C_M^i \\
    \end{align*}

    Since $\pi$ and $F$ are drawn independently, we can consider these conditions on them
    separately. For $F$, since all $C_M^i$ are distinct,
    these are once again independent fair random draws of
    binary strings of the appropriate length:
    \begin{displaymath}
        \probsub{
            F \sample (\mathcal{N} \rightarrow \bin^{l_S})
        }{
            \forall_i : F(C_M^i)[0;\abs{P^i} - n] = P_L^i \oplus C_L^i
        } = \prod_i 2^{-(\abs{P^i} - n)}
    \end{displaymath}

    For $\pi$, again given that all $P_M^i$ are distinct and all $C_M^i$ are distinct,
    we have that
    \begin{displaymath}
        \condprobsub{
            \pi \sample \Perm(\mathcal{R})
        }{
            \pi(P_M^j) = C_M^j
        }{
            \forall_{0 \leq i < j} : \pi(P_M^i) = C_M^i
        } = \frac{1}{2^n - j}
    \end{displaymath}
    (we number queries in the range $i \in [0 \ldots q-1]$) and therefore that:
    \begin{displaymath}
        \probsub{
            \pi \sample \Perm(\mathcal{R})
        }{
            \forall_i : \pi(P_M^i) = C_M^i
        } = \prod_i \frac{1}{2^n - i}
    \end{displaymath}

    Therefore:
    \begin{align*}
        {}&\condprobsub{\Omega_X}{\omega \in \comp_X(\tau)}{\omega = (K_H, ., .)} \\
        ={}& \probsub{
            \pi \sample \Perm(\mathcal{R}),
            F \sample (\mathcal{N} \rightarrow \bin^{l_S})
        }{
            \forall_i : \eta_{K_H, \pi, F, T^i}(P^i) = C^i
        } \\
        ={}& \prod_i \frac{1}{2^n - i}2^{-(\abs{P^i} - n)} \\
        \geq {}& \prod_i 2^{-\abs{P^i}} = \probsub{\Omega_Y}{\omega \in \comp_Y(\tau)} \qedhere \\
    \end{align*}
\end{proof}

\begin{lemma} \label{xyadv}
    \begin{displaymath}
        \abs{\rho_X - \rho_Y}
        \leq (\epsilon + 2^{-n})\binom{q}{2}
    \end{displaymath}
\end{lemma}

\begin{proof}
    For any transcript $\tau$ such that $\prob{Y = \tau} > 0$:
    \begin{align*}
        {}& \minone{
               \frac{\probsub{\Omega_X}{\omega \in \comp_X(\tau)}}
               {\probsub{\Omega_Y}{\omega \in \comp_Y(\tau)}}
            } \\
        ={}& \minone{
              \frac{\expsub{K_H \sample \mathcal{K}_H}{\condprobsub{\Omega_X}{\omega \in \comp_X(\tau)}{\omega = (K_H, ., .)}}}
              {\probsub{\Omega_Y}{\omega \in \comp_Y(\tau)}}} \\
        \intertext{$\minone{\expect{U}} \geq \expect{\minone{U}}$ for any real random variable $U$, therefore}
        \geq{}&
            \expsub{K_H \sample \mathcal{K}_H}{
              \minone{
              \frac{\condprobsub{\Omega_X}{\omega \in \comp_X(\tau)}{\omega = (K_H, ., .)}}
              {\probsub{\Omega_Y}{\omega \in \comp_Y(\tau)}}
              }} \\
        \geq{}&
            \probsub{K_H \sample \mathcal{K}_H}{
                \frac{\condprobsub{\Omega_X}{\omega \in \comp_X(\tau)}{\omega = (K_H, ., .)}}
                {\probsub{\Omega_Y}{\omega \in \comp_Y(\tau)}} \geq 1
            } \\
        \intertext{by \autoref{notbad}}
        \geq{}& \probsub{K_H \sample \mathcal{K}_H}{(K_H, \tau) \notin \bad} \\
    \end{align*}

    Using the H-coefficient technique:
    \begin{align*}
        {}&\abs{\rho_X - \rho_Y} \\
        \leq{}& 1 - \expsub{\tau \sample Y}{\minone{
               \frac{\probsub{\Omega_X}{\omega \in \comp_X(\tau)}}
               {\probsub{\Omega_Y}{\omega \in \comp_Y(\tau)}}
            }} \\
        \leq{}& 1 - \expsub{\tau \sample Y}{
            \probsub{K_H \sample \mathcal{K}_H}{(K_H, \tau) \notin \bad}} \\
        = {}& \probsub{\tau \sample Y, K_H \sample \mathcal{K}_H}{(K_H, \tau) \in \bad} \\
        \leq {}& \probsub{\tau \sample Y, K_H \sample \mathcal{K}_H}{(K_H, \tau) \in \badQ}
         + \probsub{\tau \sample Y, K_H \sample \mathcal{K}_H}{(K_H, \tau) \in \badR} \\
         \intertext{by \autoref{badQ} and \autoref{badR}}
        \leq {}& (\epsilon + 2^{-n})\binom{q}{2} \qedhere \\
    \end{align*}
\end{proof}

\subsection{Proof of primary claim}
\begin{proof}[Proof of \autoref{hbshadvantage}]\label{hbshproof}
    Copying the definitions of $\rho_X$, $\rho_Y$ from \autoref{rhox},
    we define
    \begin{align*}
        \rho_V \defeq{}& \probsub{K_S \sample \mathcal{K}_S}
            {A^{\HBSH_{K_S}, \HBSH_{K_S}^{-1}}\Rightarrow 1} \\
        \rho_W \defeq{}& \probsub{F \sample (\mathcal{N} \rightarrow \bin^{l_S})}
            {A^{\bar{\eta}_{F}, \bar{\eta}_{F}^{-1}}\Rightarrow 1} \\
        \rho_X \defeq{}& \probsub{\Omega_X}{A^{\Upsilon_X(\omega)} \Rightarrow 1} \\
            ={}&\probsub{K_H, \pi, F \sample \Omega_X}{A^{\eta_{K_H, \pi, F}, \eta_{K_H, \pi, F}^{-1}} \Rightarrow 1} \\
        \rho_Y \defeq{}& \probsub{\Omega_Y}{A^{\Upsilon_Y(\omega)} \Rightarrow 1} \\
            ={}&\probsub{\mathcal{E}, \mathcal{D} \sample \LP^\mathcal{T}(\mathcal{M}) \times \LP^\mathcal{T}(\mathcal{M})}{A^{\mathcal{E}, \mathcal{D}} \Rightarrow 1} \\
        \rho_Z \defeq{}& \probsub{\bm{\pi} \sample \Perm^\mathcal{T}(\mathcal{M})}
            {A^{\bm{\pi},\bm{\pi}^{-1}}\Rightarrow 1} \\
    \end{align*}

    Distinguishing
    $\rho_V$ and $\rho_W$ is distinguishing the substitution of a stream cipher
    for a random function.
    Including the key schedule, the adversary distinguishing
    $\rho_V$ and $\rho_W$ makes at most $q + 1$ queries on the stream cipher
    or random function respectively, and uses at most $\abs{K_E} + \abs{K_H} + q(l_M - n)$ bits
    of the output; by a standard substitution argument per \cite{cbcsec,concsym}
    $\abs{\rho_V - \rho_W} \leq
    \advantage{\mathrm{sc}}{S}[(q + 1, \abs{K_E} + \abs{K_H} + q(l_M - n), t')]$
    where $t' = t + \bigO{q(l_T + l_M)}$.

    The differences between $\rho_W$ and $\rho_X$ are the use of a block cipher
    in place of a random permutation, and the use of $F(\lambda)$ to determine
    $K_E$ and $K_H$. Since $F$ is a random function and $F(\lambda)$ is used
    only here, this is equivalent to choosing them at random; again by a substitution
    argument we have that $\abs{\rho_W - \rho_X} \leq \advantage{\pm \mathrm{prp}}{E}[(q, t')]$.

    $\abs{\rho_X - \rho_Y} \leq (\epsilon + 2^{-n})\binom{q}{2}$ by \autoref{xyadv}.
    Since we forbid pointless queries,
    $\abs{\rho_Y - \rho_Z} \leq 2^{-n}\binom{q}{2}$ by Halevi and Rogaway's PRP-RND lemma
    (\cite{cmc}, Appendix C, Lemma 6).

    \autoref{hbshadvantage} follows by summing these bounds:
    $\abs{\rho_V - \rho_Z} \leq
    \abs{\rho_V - \rho_W} + \abs{\rho_W - \rho_X} + \abs{\rho_X - \rho_Y} + \abs{\rho_Y - \rho_Z}$.
\end{proof}

\subbib
\end{document}
